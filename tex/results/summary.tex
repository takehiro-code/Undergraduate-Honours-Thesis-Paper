\section{Summary}
\label{sec:results/summary}

We presented the average tracking results of 12 video sequences for "all" object classes. Regression analysis was conducted to quantify the impact of QP and MSR on MOTA and formulate the relationship between MOTA and QP. One-sided t-test was conducted to identify the specific QP at which performance on compressed sequences is lower than on uncompressed sequences with 95\% confidence. The results indicate that most metrics except Precision and MOTP are consistent with our expectation. To achieve a better understanding of the average results, we inspected four individual sequences. From each video, we saw that as the image quality drops, the detector starts to not detect the target, showing the increase of FN. We also found that there are cases that make certain performance scores increase as the image quality decreases. This is because the tracking performance depends on the detector's performance. Our detector YOLOv3 sometimes incorrectly detects wrong targets, but as the image quality drops, the frequency of wrong detections decreases. We also learned that for videos with frequent occlusions, IDs decreases due to fewer detections and therefore less detected occlusions as the image quality drops. However, for videos with fewer occlusions, IDs increases due to "gap" of trajectories, and different identities would be assigned to the trajectories. Note that because SORT does not have a re-ID feature for long-term occlusions or undetection, we observed such IDs result. Finally, the impact of detection performance on tracking performance was examined. From the scatter plots of MOTA and mAP-50, we observed a linear relationship in some videos. However, some videos show a non-linear relationship such that as mAP-50 increases, MOTA increases but the rate of increase of MOTA decreases. The more thorough analysis will need to be conducted to understand this outcome.