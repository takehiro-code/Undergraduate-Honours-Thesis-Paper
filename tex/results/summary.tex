\section{Summary}
\label{sec:results/summary}

We illustrated the averaged result of 12 video samples and analyzed the individual sequences. From the averaged result of all video samples, as QP increases, the general tracking performance of MOTA, detection performance of F1, ID measure of IDF1, and the track quality performance decrease. This is consistent with the hypothesis such that the performance decreases as the QP increases and the image quality drops. However, Precision and MOTP were not consistent with the hypothesis. To support the averaged result, we inspected 4 individual sequences. From each video, we proved that as the image quality drops, the detector starts to not detect the target, proving the increase of FN. We also found that there are cases that makes the performance increases as the image quality decreases. This is because the tracking performance is dependent on the ability of the detector, and our detector YOLO v3 incorrectly detect the wrong targets, the type object that can be detected as others, but as the image quality decreases, the frequency of wrong detection decreases. Finally, we learned that for the video with the frequent occlusions, IDs decreases due to the less detections and therefore less detected occlusions as the image quality drops. For the video with the less occlusions; however, IDs increases because the detection becomes more discontinuous and different identities would be assigned to the trajectories. Note that because SORT does not have re-ID feature for the long term occlusions or undetection or occlusions, we observed such IDs result.