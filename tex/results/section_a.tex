\section{Averaged Result of Multiple Video Sequences}
\label{sec:results/section_a}

Running the experiment with all the possible combinations of QP and MSR as shown in Table \ref{tab:qp_msr_range}, we obtained the result from all the 12 video sequences or 12 samples in a statistical term. Since we have multiple video samples, we averaged each score of metrics with respect to QP and MSR. Table \ref{tab:averaged_result_all} shows mean values of each score from the different metrics over different QP and MSR. Each numerical value is a mean of 12 measured values of score. Table \ref{tab:averaged_result_all_std} shows their standard deviation values. Since each video sample differs in resolution, frame rare, number of objects and object classes, tracking performance is different. Due to the lack of video samples, the standard deviation is high among the 12 video samples. We believe that to have lower standard deviation, we need to have more video samples. As there are 20 metrics for object tracking performance and MOTA is a good indicator for the general tracking performance as explained in Chapter \ref{chap:background}, we visualized MOTA score for different QP and MSR as shown in Figure \ref{fig:averaged_result_all}. 
\begin{table}[!htbp]
  \centering
  \caption[Averaged performance results of all video samples]
  {Averaged performance results of all video samples.}

  % table for uncompressed
  \begin{subtable}[t]{\linewidth}
    \centering
    \vspace{0pt}
    \resizebox{1.0\linewidth}{!}{
    \begin{tabular}{llrrrrrrrrrrrrrrrrrrrr}
    \toprule
              QP &          MSR &    IDTP &   IDFP &    IDFN &  IDF1 &   IDP &   IDR &  Recall &  Precision &    F1 &    GT &   MT &   PT &   ML &      TP &     FP &      FN &   IDs &    FM &  MOTA &  MOTP \\
    \midrule
    Uncompressed & Uncompressed & 1762.58 & 932.83 & 2522.67 & 49.88 & 61.13 & 43.48 &   62.17 &      89.25 & 72.08 & 12.25 & 4.67 & 3.25 & 4.33 & 2249.75 & 441.67 & 2035.50 & 24.08 & 35.17 & 54.49 & 82.03 \\
    \bottomrule
    \end{tabular}}
    \caption{Mean values for the uncompressed sequence}
  \end{subtable}
 
 
 
  
  % table for msr=8
  \begin{subtable}[t]{\linewidth}
    \centering
    \resizebox{1.0\linewidth}{!}{
    \begin{tabular}{rrrrrrrrrrrrrrrrrrrrrr}
    \toprule
     QP &  MSR &    IDTP &   IDFP &    IDFN &  IDF1 &   IDP &   IDR &  Recall &  Precision &    F1 &    GT &   MT &   PT &   ML &      TP &     FP &      FN &   IDs &    FM &  MOTA &  MOTP \\
    \midrule
     18 &    8 & 1770.08 & 947.50 & 2515.17 & 50.60 & 61.55 & 44.26 &   62.18 &      88.58 & 71.92 & 12.25 & 4.50 & 3.17 & 4.58 & 2246.75 & 466.83 & 2038.50 & 24.17 & 36.00 & 53.97 & 82.11 \\
     22 &    8 & 1780.83 & 931.92 & 2504.42 & 50.46 & 61.66 & 43.96 &   61.62 &      88.85 & 71.65 & 12.25 & 4.58 & 3.17 & 4.50 & 2250.00 & 458.75 & 2035.25 & 24.92 & 35.58 & 53.70 & 82.10 \\
     26 &    8 & 1751.67 & 874.17 & 2533.58 & 50.41 & 62.63 & 43.28 &   60.49 &      89.73 & 71.22 & 12.25 & 4.33 & 3.17 & 4.75 & 2192.92 & 428.92 & 2092.33 & 24.33 & 35.50 & 53.24 & 82.28 \\
     30 &    8 & 1731.50 & 960.08 & 2553.75 & 49.88 & 62.12 & 42.65 &   60.21 &      89.83 & 71.07 & 12.25 & 4.33 & 3.08 & 4.83 & 2234.42 & 453.17 & 2050.83 & 23.67 & 35.00 & 53.13 & 82.16 \\
     34 &    8 & 1653.92 & 855.08 & 2631.33 & 48.14 & 62.23 & 39.93 &   57.12 &      90.61 & 69.26 & 12.25 & 4.08 & 3.00 & 5.17 & 2092.58 & 412.42 & 2192.67 & 21.75 & 32.25 & 50.64 & 82.42 \\
     38 &    8 & 1488.50 & 781.08 & 2796.75 & 44.56 & 61.38 & 35.62 &   52.70 &      91.52 & 65.97 & 12.25 & 3.42 & 3.17 & 5.67 & 1912.42 & 353.17 & 2372.83 & 21.92 & 30.67 & 47.15 & 82.29 \\
     42 &    8 & 1399.25 & 733.83 & 2886.00 & 40.92 & 60.67 & 32.08 &   46.86 &      91.35 & 60.24 & 12.25 & 3.08 & 3.25 & 5.92 & 1805.67 & 323.42 & 2479.58 & 20.83 & 33.50 & 41.48 & 81.42 \\
     46 &    8 & 1046.50 & 673.33 & 3238.75 & 30.48 & 57.12 & 22.40 &   33.09 &      87.46 & 45.48 & 12.25 & 1.75 & 3.08 & 7.42 & 1358.33 & 357.50 & 2926.92 & 18.00 & 28.25 & 27.64 & 80.47 \\
    \bottomrule
    \end{tabular}}
    \caption{Mean values for MSR = 8}
  \end{subtable}
 
 
  
  % table for msr=16
  \begin{subtable}[t]{\linewidth}
    \centering
    \resizebox{1.0\linewidth}{!}{
    \begin{tabular}{rrrrrrrrrrrrrrrrrrrrrr}
    \toprule
     QP &  MSR &    IDTP &   IDFP &    IDFN &  IDF1 &   IDP &   IDR &  Recall &  Precision &    F1 &    GT &   MT &   PT &   ML &      TP &     FP &      FN &   IDs &    FM &  MOTA &  MOTP \\
    \midrule
     18 &   16 & 1749.25 & 964.25 & 2536.00 & 50.48 & 61.51 & 44.07 &   62.04 &      88.73 & 71.88 & 12.25 & 4.50 & 3.33 & 4.42 & 2248.00 & 461.50 & 2037.25 & 25.08 & 35.25 & 54.03 & 82.08 \\
     22 &   16 & 1790.08 & 932.17 & 2495.17 & 50.40 & 61.45 & 43.96 &   61.39 &      88.42 & 71.38 & 12.25 & 4.58 & 3.17 & 4.50 & 2242.17 & 476.08 & 2043.08 & 24.33 & 35.75 & 53.36 & 82.16 \\
     26 &   16 & 1728.75 & 886.25 & 2556.50 & 50.48 & 62.82 & 43.28 &   60.43 &      89.89 & 71.21 & 12.25 & 4.33 & 3.42 & 4.50 & 2190.50 & 420.50 & 2094.75 & 23.58 & 34.08 & 53.36 & 82.40 \\
     30 &   16 & 1728.33 & 954.08 & 2556.92 & 48.91 & 61.02 & 41.83 &   60.04 &      90.00 & 71.10 & 12.25 & 4.17 & 3.42 & 4.67 & 2225.42 & 453.00 & 2059.83 & 23.17 & 34.25 & 53.21 & 82.26 \\
     34 &   16 & 1604.67 & 906.00 & 2680.58 & 46.94 & 60.73 & 38.90 &   56.89 &      90.40 & 69.10 & 12.25 & 3.83 & 3.17 & 5.25 & 2087.08 & 419.58 & 2198.17 & 21.67 & 32.08 & 50.49 & 82.29 \\
     38 &   16 & 1496.58 & 791.58 & 2788.67 & 45.47 & 62.23 & 36.46 &   53.13 &      91.95 & 66.47 & 12.25 & 3.58 & 3.08 & 5.58 & 1934.58 & 349.58 & 2350.67 & 23.17 & 31.83 & 47.81 & 82.04 \\
     42 &   16 & 1379.33 & 786.17 & 2905.92 & 39.76 & 58.32 & 31.38 &   47.28 &      90.98 & 60.56 & 12.25 & 2.92 & 3.25 & 6.08 & 1809.33 & 352.17 & 2475.92 & 20.33 & 33.58 & 41.72 & 81.33 \\
     46 &   16 & 1078.17 & 613.00 & 3207.08 & 32.42 & 61.75 & 23.68 &   33.48 &      88.70 & 45.99 & 12.25 & 1.92 & 2.83 & 7.50 & 1368.58 & 318.58 & 2916.67 & 18.42 & 28.58 & 27.67 & 79.77 \\
    \bottomrule
    \end{tabular}}
    \caption{Mean values for MSR = 16}
  \end{subtable}
 
 
 
  % table for msr=32
  \begin{subtable}[t]{\linewidth}
    \centering
    \resizebox{1.0\linewidth}{!}{
    \begin{tabular}{rrrrrrrrrrrrrrrrrrrrrr}
    \toprule
     QP &  MSR &    IDTP &   IDFP &    IDFN &  IDF1 &   IDP &   IDR &  Recall &  Precision &    F1 &    GT &   MT &   PT &   ML &      TP &     FP &      FN &   IDs &    FM &  MOTA &  MOTP \\
    \midrule
     18 &   32 & 1768.58 & 939.58 & 2516.67 & 50.67 & 61.70 & 44.31 &   61.98 &      88.62 & 71.78 & 12.25 & 4.58 & 3.25 & 4.42 & 2245.33 & 458.83 & 2039.92 & 25.00 & 35.58 & 53.80 & 82.08 \\
     22 &   32 & 1791.42 & 920.08 & 2493.83 & 50.38 & 61.48 & 43.92 &   61.49 &      88.46 & 71.42 & 12.25 & 4.58 & 3.25 & 4.42 & 2254.00 & 453.50 & 2031.25 & 25.00 & 36.42 & 53.15 & 82.18 \\
     26 &   32 & 1724.42 & 939.08 & 2560.83 & 49.68 & 61.44 & 42.82 &   60.80 &      89.44 & 71.32 & 12.25 & 4.42 & 3.08 & 4.75 & 2204.33 & 455.17 & 2080.92 & 24.25 & 35.75 & 53.41 & 82.34 \\
     30 &   32 & 1697.08 & 950.92 & 2588.17 & 49.14 & 61.34 & 41.91 &   59.79 &      89.85 & 70.91 & 12.25 & 4.17 & 3.42 & 4.67 & 2196.08 & 447.92 & 2089.17 & 24.08 & 36.58 & 52.83 & 82.25 \\
     34 &   32 & 1628.42 & 865.50 & 2656.83 & 48.13 & 61.98 & 39.98 &   57.10 &      90.70 & 69.32 & 12.25 & 4.00 & 3.00 & 5.25 & 2096.75 & 393.17 & 2188.50 & 21.42 & 31.58 & 50.83 & 82.38 \\
     38 &   32 & 1512.50 & 778.75 & 2772.75 & 45.10 & 61.90 & 36.13 &   53.20 &      91.29 & 66.31 & 12.25 & 3.58 & 3.08 & 5.58 & 1934.67 & 352.58 & 2350.58 & 20.75 & 31.08 & 47.57 & 81.98 \\
     42 &   32 & 1367.25 & 779.58 & 2918.00 & 39.92 & 58.70 & 31.38 &   47.28 &      90.61 & 60.48 & 12.25 & 3.00 & 3.42 & 5.83 & 1808.25 & 334.58 & 2477.00 & 19.67 & 32.25 & 41.48 & 81.16 \\
     46 &   32 & 1018.67 & 693.75 & 3266.58 & 30.47 & 57.32 & 22.52 &   33.53 &      88.54 & 45.93 & 12.25 & 1.75 & 3.08 & 7.42 & 1363.75 & 344.67 & 2921.50 & 19.25 & 28.25 & 27.97 & 81.11 \\
    \bottomrule
    \end{tabular}}
    \caption{Mean values for MSR = 32}
  \end{subtable}
  
  
  % table for msr=64
  \begin{subtable}[t]{\linewidth}
    \centering
    \resizebox{1.0\linewidth}{!}{
    \begin{tabular}{rrrrrrrrrrrrrrrrrrrrrr}
    \toprule
     QP &  MSR &    IDTP &   IDFP &    IDFN &  IDF1 &   IDP &   IDR &  Recall &  Precision &    F1 &    GT &   MT &   PT &   ML &      TP &     FP &      FN &   IDs &    FM &  MOTA &  MOTP \\
    \midrule
     18 &   64 & 1756.00 & 964.33 & 2529.25 & 50.10 & 61.04 & 43.73 &   62.17 &      88.82 & 72.02 & 12.25 & 4.67 & 3.00 & 4.58 & 2252.83 & 463.50 & 2032.42 & 24.75 & 36.75 & 54.24 & 82.06 \\
     22 &   64 & 1768.75 & 944.75 & 2516.50 & 50.68 & 61.77 & 44.22 &   61.56 &      88.53 & 71.48 & 12.25 & 4.42 & 3.33 & 4.50 & 2244.83 & 464.67 & 2040.42 & 24.75 & 36.75 & 53.33 & 82.12 \\
     26 &   64 & 1722.25 & 906.50 & 2563.00 & 49.46 & 61.60 & 42.30 &   60.18 &      89.24 & 70.90 & 12.25 & 4.33 & 3.25 & 4.67 & 2184.42 & 440.33 & 2100.83 & 24.33 & 34.50 & 52.62 & 82.37 \\
     30 &   64 & 1731.25 & 949.00 & 2554.00 & 49.44 & 61.59 & 42.27 &   59.80 &      89.16 & 70.64 & 12.25 & 4.33 & 3.08 & 4.83 & 2214.75 & 461.50 & 2070.50 & 23.83 & 34.50 & 52.47 & 82.28 \\
     34 &   64 & 1615.58 & 880.42 & 2669.67 & 46.90 & 60.65 & 38.91 &   57.32 &      91.40 & 69.69 & 12.25 & 3.83 & 3.25 & 5.17 & 2102.25 & 389.75 & 2183.00 & 20.92 & 32.17 & 51.51 & 82.42 \\
     38 &   64 & 1508.92 & 765.42 & 2776.33 & 46.25 & 63.23 & 37.09 &   52.56 &      90.69 & 65.69 & 12.25 & 3.50 & 3.08 & 5.67 & 1917.75 & 352.58 & 2367.50 & 20.33 & 31.83 & 46.48 & 81.96 \\
     42 &   64 & 1404.50 & 744.42 & 2880.75 & 42.43 & 62.55 & 33.18 &   47.31 &      90.25 & 60.65 & 12.25 & 3.00 & 3.33 & 5.92 & 1805.17 & 339.75 & 2480.08 & 18.83 & 32.75 & 41.57 & 81.38 \\
     46 &   64 &  995.00 & 729.42 & 3290.25 & 29.98 & 56.74 & 22.02 &   33.18 &      87.92 & 45.43 & 12.25 & 1.75 & 3.08 & 7.42 & 1364.33 & 356.08 & 2920.92 & 19.25 & 29.17 & 27.29 & 80.75 \\
    \bottomrule
    \end{tabular}}
    \caption{Mean values for MSR = 64}
  \end{subtable}

  
\label{tab:averaged_result_all}  
\end{table}
\begin{table}[!hb]
  \centering
  \caption[Standard deviation of performance results of all video samples]
  {Standard deviation of performance results across all video sequences.}

  % table for uncompressed
  \begin{subtable}[t]{\linewidth}
    \centering
    \vspace{0pt}
    \resizebox{1.0\linewidth}{!}{
    \begin{tabular}{llrrrrrrrrrrrrrrrrrrrr}
    \toprule
              QP &          MSR &    IDTP &   IDFP &    IDFN &  IDF1 &   IDP &   IDR &  Rcll &  Prcn &    F1 &   GT &   MT &   PT &   ML &      TP &     FP &      FN &   IDs &    FM &  MOTA &  MOTP \\
    \midrule
    Uncompressed & Uncompressed & 2064.78 & 920.95 & 3363.60 & 21.39 & 19.91 & 23.64 & 20.12 & 10.80 & 16.29 & 7.70 & 3.52 & 3.72 & 5.42 & 2050.74 & 731.69 & 3253.93 & 28.13 & 36.08 & 21.79 &  0.05 \\
    \bottomrule
    \end{tabular}}
    \caption{Standard Deviation for the Uncompressed Sequence}
  \end{subtable}
 
 
 
  
  % table for msr=8
  \begin{subtable}[t]{\linewidth}
    \centering
    \resizebox{1.0\linewidth}{!}{
    \begin{tabular}{rrrrrrrrrrrrrrrrrrrrrr}
    \toprule
     QP &  MSR &    IDTP &   IDFP &    IDFN &  IDF1 &   IDP &   IDR &  Rcll &  Prcn &    F1 &   GT &   MT &   PT &   ML &      TP &     FP &      FN &   IDs &    FM &  MOTA &  MOTP \\
    \midrule
     18 &    8 & 2060.04 & 979.45 & 3377.44 & 22.75 & 21.09 & 24.90 & 19.98 & 11.66 & 16.32 & 7.70 & 3.61 & 3.43 & 5.30 & 2058.07 & 778.56 & 3241.03 & 27.89 & 36.76 & 22.21 &  0.05 \\
     22 &    8 & 2090.58 & 933.34 & 3335.78 & 22.71 & 20.92 & 24.90 & 19.95 & 11.46 & 16.29 & 7.70 & 3.55 & 3.61 & 5.32 & 2072.84 & 756.57 & 3219.86 & 27.76 & 35.89 & 22.36 &  0.05 \\
     26 &    8 & 2030.44 & 843.94 & 3236.65 & 21.82 & 20.84 & 23.07 & 18.51 & 11.00 & 15.54 & 7.70 & 3.47 & 3.46 & 5.51 & 1994.64 & 714.86 & 3185.43 & 27.07 & 35.28 & 20.68 &  0.05 \\
     30 &    8 & 1980.42 & 973.34 & 3268.86 & 20.86 & 20.46 & 22.09 & 18.64 & 10.38 & 15.17 & 7.70 & 3.70 & 3.34 & 5.18 & 2085.29 & 770.80 & 3132.23 & 26.29 & 33.79 & 20.91 &  0.05 \\
     34 &    8 & 1879.30 & 868.16 & 3230.65 & 19.17 & 20.30 & 18.72 & 15.42 & 10.75 & 13.68 & 7.70 & 3.50 & 3.49 & 5.27 & 1884.81 & 702.41 & 3113.31 & 24.99 & 33.42 & 17.89 &  0.05 \\
     38 &    8 & 1702.59 & 757.26 & 3326.29 & 17.56 & 20.63 & 16.08 & 14.58 &  9.68 & 13.15 & 7.70 & 3.12 & 3.38 & 5.28 & 1707.19 & 594.40 & 3227.73 & 25.31 & 33.33 & 15.75 &  0.05 \\
     42 &    8 & 1706.42 & 726.48 & 3431.96 & 17.76 & 18.58 & 17.50 & 17.53 &  9.54 & 15.68 & 7.70 & 3.23 & 3.19 & 5.26 & 1798.04 & 547.93 & 3299.26 & 22.50 & 29.19 & 17.55 &  0.05 \\
     46 &    8 & 1392.69 & 859.76 & 3567.74 & 16.97 & 17.51 & 15.36 & 18.60 & 12.29 & 19.96 & 7.70 & 2.09 & 2.47 & 5.53 & 1530.73 & 732.53 & 3438.87 & 16.26 & 24.77 & 18.38 &  0.04 \\
    \bottomrule
    \end{tabular}}
    \caption{Standard Deviation at MSR = 8}
  \end{subtable}
 
 
  
  % table for msr=16
  \begin{subtable}[t]{\linewidth}
    \centering
    \resizebox{1.0\linewidth}{!}{
    \begin{tabular}{rrrrrrrrrrrrrrrrrrrrrr}
    \toprule
     QP &  MSR &    IDTP &    IDFP &    IDFN &  IDF1 &   IDP &   IDR &  Rcll &  Prcn &    F1 &   GT &   MT &   PT &   ML &      TP &     FP &      FN &   IDs &    FM &  MOTA &  MOTP \\
    \midrule
     18 &   16 & 2011.62 & 1027.53 & 3447.42 & 22.45 & 20.88 & 24.53 & 20.15 & 11.59 & 16.48 & 7.70 & 3.61 & 3.65 & 5.28 & 2055.22 & 768.69 & 3246.40 & 28.47 & 36.29 & 22.45 &  0.05 \\
     22 &   16 & 2113.02 &  934.99 & 3310.19 & 23.44 & 21.89 & 25.45 & 20.20 & 11.64 & 16.63 & 7.70 & 3.55 & 3.61 & 5.42 & 2063.34 & 786.66 & 3233.91 & 28.68 & 36.85 & 22.93 &  0.05 \\
     26 &   16 & 1977.59 &  886.72 & 3304.55 & 21.98 & 21.12 & 23.09 & 18.54 & 10.99 & 15.57 & 7.70 & 3.52 & 3.78 & 5.21 & 1993.54 & 694.27 & 3178.99 & 27.47 & 34.43 & 20.66 &  0.05 \\
     30 &   16 & 1991.49 &  997.00 & 3336.38 & 23.54 & 23.96 & 24.26 & 18.50 & 10.96 & 15.35 & 7.70 & 3.61 & 3.92 & 5.28 & 2067.08 & 745.75 & 3142.65 & 26.21 & 33.11 & 21.30 &  0.05 \\
     34 &   16 & 1822.60 &  933.65 & 3313.62 & 19.61 & 20.81 & 19.14 & 15.28 & 10.60 & 13.68 & 7.70 & 3.35 & 3.43 & 5.24 & 1883.27 & 693.22 & 3115.29 & 26.03 & 34.02 & 17.93 &  0.05 \\
     38 &   16 & 1674.77 &  814.20 & 3399.62 & 18.00 & 20.17 & 16.87 & 14.47 &  9.65 & 13.09 & 7.70 & 3.09 & 3.29 & 5.32 & 1726.97 & 600.13 & 3232.38 & 25.45 & 32.75 & 15.83 &  0.05 \\
     42 &   16 & 1699.11 &  782.77 & 3465.90 & 17.88 & 18.58 & 17.87 & 17.71 & 10.17 & 15.78 & 7.70 & 2.91 & 2.86 & 5.38 & 1781.06 & 629.23 & 3354.80 & 22.51 & 29.91 & 17.94 &  0.05 \\
     46 &   16 & 1385.56 &  796.19 & 3581.80 & 16.58 & 19.46 & 15.00 & 18.15 & 14.03 & 19.15 & 7.70 & 2.23 & 2.33 & 5.76 & 1514.79 & 636.69 & 3450.12 & 17.41 & 24.10 & 18.67 &  0.04 \\
    \bottomrule
    \end{tabular}}
    \caption{Standard Deviation at MSR = 16}
  \end{subtable}
 
 
 
  % table for msr=32
  \begin{subtable}[t]{\linewidth}
    \centering
    \resizebox{1.0\linewidth}{!}{
    \begin{tabular}{rrrrrrrrrrrrrrrrrrrrrr}
    \toprule
     QP &  MSR &    IDTP &   IDFP &    IDFN &  IDF1 &   IDP &   IDR &  Rcll &  Prcn &    F1 &   GT &   MT &   PT &   ML &      TP &     FP &      FN &   IDs &    FM &  MOTA &  MOTP \\
    \midrule
     18 &   32 & 2043.60 & 978.25 & 3401.33 & 23.06 & 21.33 & 25.21 & 20.14 & 11.93 & 16.49 & 7.70 & 3.55 & 3.72 & 5.37 & 2052.97 & 772.83 & 3251.43 & 28.59 & 35.81 & 22.60 &  0.05 \\
     22 &   32 & 2120.42 & 901.69 & 3298.39 & 23.13 & 21.77 & 25.10 & 19.87 & 11.67 & 16.22 & 7.70 & 3.55 & 3.70 & 5.21 & 2077.58 & 742.03 & 3204.20 & 28.72 & 35.76 & 22.54 &  0.05 \\
     26 &   32 & 1996.29 & 974.00 & 3345.79 & 21.15 & 19.81 & 22.76 & 19.07 & 11.20 & 15.89 & 7.70 & 3.45 & 3.48 & 5.41 & 2017.05 & 773.88 & 3200.13 & 27.29 & 35.99 & 21.35 &  0.05 \\
     30 &   32 & 1928.22 & 979.63 & 3326.39 & 22.31 & 22.40 & 22.85 & 17.73 & 10.88 & 15.02 & 7.70 & 3.41 & 3.42 & 5.26 & 2004.73 & 736.84 & 3133.16 & 26.69 & 34.16 & 20.50 &  0.05 \\
     34 &   32 & 1825.27 & 900.55 & 3318.31 & 21.58 & 22.86 & 20.77 & 15.35 & 10.10 & 13.65 & 7.70 & 3.52 & 3.49 & 5.24 & 1890.05 & 641.81 & 3105.59 & 25.81 & 33.10 & 17.81 &  0.05 \\
     38 &   32 & 1685.45 & 781.39 & 3360.86 & 18.07 & 20.98 & 16.68 & 14.66 &  9.34 & 13.01 & 7.70 & 3.09 & 3.29 & 5.32 & 1727.57 & 581.72 & 3225.25 & 23.83 & 32.00 & 15.73 &  0.05 \\
     42 &   32 & 1648.36 & 734.24 & 3389.41 & 15.87 & 16.67 & 15.80 & 17.39 & 10.31 & 15.50 & 7.70 & 3.05 & 3.15 & 5.41 & 1790.08 & 590.90 & 3320.16 & 21.98 & 30.56 & 17.76 &  0.05 \\
     46 &   32 & 1349.58 & 853.75 & 3656.11 & 16.71 & 16.56 & 15.46 & 18.52 & 12.12 & 19.61 & 7.70 & 2.30 & 2.91 & 5.79 & 1524.22 & 654.05 & 3456.88 & 16.90 & 23.08 & 17.82 &  0.04 \\
    \bottomrule
    \end{tabular}}
    \caption{Standard Deviation at MSR = 32}
  \end{subtable}
  
  
  % table for msr=64
  \begin{subtable}[t]{\linewidth}
    \centering
    \resizebox{1.0\linewidth}{!}{
    \begin{tabular}{rrrrrrrrrrrrrrrrrrrrrr}
    \toprule
     QP &  MSR &    IDTP &   IDFP &    IDFN &  IDF1 &   IDP &   IDR &  Rcll &  Prcn &    F1 &   GT &   MT &   PT &   ML &      TP &     FP &      FN &   IDs &    FM &  MOTA &  MOTP \\
    \midrule
     18 &   64 & 2046.04 & 982.64 & 3395.07 & 21.92 & 20.27 & 24.08 & 20.05 & 11.46 & 16.41 & 7.70 & 3.52 & 3.46 & 5.48 & 2059.74 & 767.82 & 3239.35 & 28.20 & 36.91 & 22.29 &  0.05 \\
     22 &   64 & 2065.76 & 985.02 & 3377.98 & 23.29 & 21.89 & 25.24 & 19.86 & 11.62 & 16.18 & 7.70 & 3.63 & 3.65 & 5.35 & 2072.95 & 771.13 & 3214.79 & 28.75 & 37.09 & 22.30 &  0.05 \\
     26 &   64 & 1973.12 & 860.48 & 3209.07 & 20.43 & 20.00 & 21.36 & 18.20 & 11.00 & 15.46 & 7.70 & 3.50 & 3.62 & 5.14 & 1962.76 & 739.29 & 3167.65 & 27.64 & 35.15 & 20.74 &  0.05 \\
     30 &   64 & 2017.15 & 914.00 & 3239.18 & 21.74 & 21.74 & 22.71 & 18.74 & 11.22 & 15.66 & 7.70 & 3.68 & 3.42 & 5.32 & 2060.76 & 748.33 & 3164.25 & 27.02 & 33.51 & 21.70 &  0.05 \\
     34 &   64 & 1819.18 & 913.01 & 3317.98 & 21.57 & 22.88 & 20.80 & 15.38 & 10.22 & 13.66 & 7.70 & 3.35 & 3.55 & 5.27 & 1884.95 & 663.61 & 3108.10 & 23.93 & 33.79 & 17.90 &  0.05 \\
     38 &   64 & 1669.71 & 764.32 & 3358.91 & 17.47 & 19.86 & 16.16 & 13.86 &  9.62 & 12.50 & 7.70 & 2.91 & 3.34 & 5.28 & 1698.78 & 583.67 & 3238.10 & 23.42 & 30.96 & 14.96 &  0.05 \\
     42 &   64 & 1621.58 & 730.11 & 3428.76 & 16.23 & 18.02 & 16.08 & 16.95 &  9.65 & 15.01 & 7.70 & 2.95 & 3.06 & 5.37 & 1764.65 & 572.65 & 3350.99 & 19.69 & 29.51 & 17.48 &  0.05 \\
     46 &   64 & 1288.47 & 945.93 & 3654.56 & 16.93 & 20.13 & 15.05 & 18.78 & 13.81 & 19.93 & 7.70 & 2.49 & 2.75 & 5.74 & 1544.48 & 696.43 & 3445.22 & 18.40 & 25.99 & 18.86 &  0.04 \\
    \bottomrule
    \end{tabular}}
    \caption{Standard Deviation at MSR = 64}
  \end{subtable}

  
\label{tab:averaged_result_all_std}  
\end{table}
\begin{figure}[!tb]
  \centering
  \includegraphics[width=1.0\linewidth]{img/averaged_result_all_qp.pdf}
  \caption[Averaged result of MOTA score from all video samples at different QP]
  {Average MOTA score across all video samples at different QP.}
  \label{fig:averaged_result_all_qp}
\end{figure}
As you can see from the plot, the uncompressed video sequences achieved the highest performance of MOTA score. For the compressed sequences, MOTA score is lower than the uncompressed result and the higher the QP, the lower the MOTA score. Not only MOTA, we observed that the performance score of most of the metrics decrease as QP increases except IDP, Precision, and MOTP. According to the hypothesis stated in Chapter \ref{chap:background/section_e}, the higher the QP, the lower the bitrate and pixel rate, so we expected the tracking performance to be lower, and our QP result from the most metrics are consistent with the hypothesis. However, for the MSR, we do not see significant differences. As another visual representation, Figure \ref{fig:averaged_result_all_msr} shows MOTA score on MSR as horizontal axis and we observed no significant differences on different MSR.
\begin{figure}[htb]
  \centering
  \includegraphics[width=1.0\linewidth]{img/averaged_result_all_msr.pdf}
  \caption[Averaged Result of All Video Samples with All Object Classes 2]{
    
  }
  \label{fig:averaged_result_all_msr}
\end{figure}

To prove the significant impact of QP and MSR on the each metrics score, we conducted a linear regression analysis. Since we have 2 continuous independent variables of QP and MSR on each of the continuous dependent variable of performance metrics score, we applied multiple linear regression on the data shown in Table \ref{tab:averaged_result_all}. We can represent the regression model as,
\begin{equation}
Score = \mathit{c_0} + \mathit{c_1} \cdot QP + \mathit{c_2} \cdot MSR + \mathit{c_3} \cdot QP \cdot MSR,
\end{equation}
where $\mathit{c_0}$ is the intercept, $\mathit{c_1}$ and $\mathit{c_2}$ are the coefficients of independent variables QP and MSR, and $\mathit{c_3}$ is the coefficient of the interaction term of QP and MSR. Score is each of the performance value from different metrics. The error term is neglected. Note that we included the interaction term to see if MSR or QP depends on each other. Table \ref{tab:regression} shows the result from the multiple linear regression analysis.
\begin{table}[!tb]
    \centering
    \caption[Regression analysis result of the MOTA score for "all" object classes across all video sequences]
    {Regression analysis result of the MOTA score for "all" object classes across all video sequences.}
    \resizebox{0.25\linewidth}{!}{
\begin{tabular}{|c|c|}
\hline
$\beta_0$          & -4.32 \\
\hline
$\beta_1$          & -2.97 \\
\hline
$\beta_2$          & $< |10^{-2}|$ \\
\hline
$\beta_3$          & $< |10^{-2}|$ \\
\hline
p-value($\beta_0$) &  $< |10^{-6}|$ \\
\hline
p-value($\beta_1$) &  $< |10^{-9}|$ \\
\hline
p-value($\beta_2$) &  0.83 \\
\hline
p-value($\beta_3$) &  0.77 \\
\hline
\end{tabular}
    }
    \label{tab:regression}
\end{table}
To see if the independent variable is statistically significant on the dependent variable, we conduct a hypothesis test at a significance level of 0.05. The null hypothesis is the case the coefficient is 0 ($H_0: c_i = 0$) and the alternative hypothesis is the case the coefficient is not zero ($H_1: c_i \neq 0$). From the table, the p-value of QP is less than 0.05 except IDP, precision, and PT, while all of p-value of MSR and the interaction term of $QP \cdot MSR$ is greater than 0.05. This shows that we reject the null hypothesis on QP, so QP has a significant impact on most metrics at a 95\% confidence. However, we fail to reject the null hypothesis for MSR and the interaction term of QP. Hence, it is insufficient to prove that MSR and the interaction term are statistically significant on each metrics. We estimated that MSR does not impact on the object tracking performance and QP and MSR does not depend on each other, and our hypothesis for MSR as explained in Chapter \ref{sec:background/section_e} is inconsistent with our results. Thus, we will limit our study on QP only.

We just showed the averaged result from the multiple video sequences, but we will justify further on the individual sequences and will find the reasoning of why the performance score of IDP, Precision, and MOTP do not decrease in the next section.

