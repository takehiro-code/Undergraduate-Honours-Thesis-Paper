\section{Results on Multiple Video Sequences}
\label{sec:results/section_a}

Running the experiment with all the possible combinations of QP and MSR as shown in Table \ref{tab:qp_msr_range}, we obtained the result from all 12 video sequences. For each pair of QP and MSR, we averaged each score across all 12 video sequences. As there are 20 metrics for evaluating object tracking performance, and MOTA is a good indicator of the overall tracking performance as explained in Section \ref{sec:background/section_d}, we focused our study on MOTA. In the following analysis, we visualized the data and conducted a regression analysis and t-test to quantify the results. The methodologies of statistical analysis we employed are according to the textbook \cite{kutner_applied_2005}.

% ----------------------------------------------------------
% Visualization
% ----------------------------------------------------------

\subsection{Visualization of Results across All Video Sequences}
\label{subsec:/results/section_a/visualization}
We visualized the MOTA score for "all" object classes across all video sequences at different QP and MSR as shown in Figure \ref{fig:averaged_result_all_qp}.
\begin{figure}[!tb]
  \centering
  \includegraphics[width=1.0\linewidth]{img/averaged_result_all_qp.pdf}
  \caption[Averaged result of MOTA score from all video samples at different QP]
  {Average MOTA score across all video samples at different QP.}
  \label{fig:averaged_result_all_qp}
\end{figure}
As can be seen from the plot, the uncompressed video sequences achieved the highest MOTA score on average. For the compressed sequences, the average MOTA score is lower than the uncompressed result, and the higher the QP, the lower the MOTA score. Not only MOTA, we observed that the performance scores of most of the other metrics decrease as QP increases. Although the scores at different MSR are shown, we do not see any significant differences between MOTA scores at different MSR values from this plot.

Figure \ref{fig:averaged_result_all_multiplots_qp} shows the visualization of all the metrics\footnote{The metric GT is omitted from the visualization since it is a constant value, and the value can be confirmed from the table.} over different QP and MSR, and Table \ref{tab:averaged_result_all} tabulates their values.
\begin{figure}[!tb]
  \centering
  \includegraphics[width=1.0\linewidth]{img/averaged_all_multiplots_qp.pdf}
  \caption[Visualization of the average performance results at different QP of all video samples]
  {Visualization of the average performance results at different QP across all video samples}
  \label{fig:averaged_result_all_multiplots_qp}
\end{figure}
\begin{table}[!htbp]
  \centering
  \caption[Averaged performance results of all video samples]
  {Averaged performance results of all video samples.}

  % table for uncompressed
  \begin{subtable}[t]{\linewidth}
    \centering
    \vspace{0pt}
    \resizebox{1.0\linewidth}{!}{
    \begin{tabular}{llrrrrrrrrrrrrrrrrrrrr}
    \toprule
              QP &          MSR &    IDTP &   IDFP &    IDFN &  IDF1 &   IDP &   IDR &  Recall &  Precision &    F1 &    GT &   MT &   PT &   ML &      TP &     FP &      FN &   IDs &    FM &  MOTA &  MOTP \\
    \midrule
    Uncompressed & Uncompressed & 1762.58 & 932.83 & 2522.67 & 49.88 & 61.13 & 43.48 &   62.17 &      89.25 & 72.08 & 12.25 & 4.67 & 3.25 & 4.33 & 2249.75 & 441.67 & 2035.50 & 24.08 & 35.17 & 54.49 & 82.03 \\
    \bottomrule
    \end{tabular}}
    \caption{Mean values for the uncompressed sequence}
  \end{subtable}
 
 
 
  
  % table for msr=8
  \begin{subtable}[t]{\linewidth}
    \centering
    \resizebox{1.0\linewidth}{!}{
    \begin{tabular}{rrrrrrrrrrrrrrrrrrrrrr}
    \toprule
     QP &  MSR &    IDTP &   IDFP &    IDFN &  IDF1 &   IDP &   IDR &  Recall &  Precision &    F1 &    GT &   MT &   PT &   ML &      TP &     FP &      FN &   IDs &    FM &  MOTA &  MOTP \\
    \midrule
     18 &    8 & 1770.08 & 947.50 & 2515.17 & 50.60 & 61.55 & 44.26 &   62.18 &      88.58 & 71.92 & 12.25 & 4.50 & 3.17 & 4.58 & 2246.75 & 466.83 & 2038.50 & 24.17 & 36.00 & 53.97 & 82.11 \\
     22 &    8 & 1780.83 & 931.92 & 2504.42 & 50.46 & 61.66 & 43.96 &   61.62 &      88.85 & 71.65 & 12.25 & 4.58 & 3.17 & 4.50 & 2250.00 & 458.75 & 2035.25 & 24.92 & 35.58 & 53.70 & 82.10 \\
     26 &    8 & 1751.67 & 874.17 & 2533.58 & 50.41 & 62.63 & 43.28 &   60.49 &      89.73 & 71.22 & 12.25 & 4.33 & 3.17 & 4.75 & 2192.92 & 428.92 & 2092.33 & 24.33 & 35.50 & 53.24 & 82.28 \\
     30 &    8 & 1731.50 & 960.08 & 2553.75 & 49.88 & 62.12 & 42.65 &   60.21 &      89.83 & 71.07 & 12.25 & 4.33 & 3.08 & 4.83 & 2234.42 & 453.17 & 2050.83 & 23.67 & 35.00 & 53.13 & 82.16 \\
     34 &    8 & 1653.92 & 855.08 & 2631.33 & 48.14 & 62.23 & 39.93 &   57.12 &      90.61 & 69.26 & 12.25 & 4.08 & 3.00 & 5.17 & 2092.58 & 412.42 & 2192.67 & 21.75 & 32.25 & 50.64 & 82.42 \\
     38 &    8 & 1488.50 & 781.08 & 2796.75 & 44.56 & 61.38 & 35.62 &   52.70 &      91.52 & 65.97 & 12.25 & 3.42 & 3.17 & 5.67 & 1912.42 & 353.17 & 2372.83 & 21.92 & 30.67 & 47.15 & 82.29 \\
     42 &    8 & 1399.25 & 733.83 & 2886.00 & 40.92 & 60.67 & 32.08 &   46.86 &      91.35 & 60.24 & 12.25 & 3.08 & 3.25 & 5.92 & 1805.67 & 323.42 & 2479.58 & 20.83 & 33.50 & 41.48 & 81.42 \\
     46 &    8 & 1046.50 & 673.33 & 3238.75 & 30.48 & 57.12 & 22.40 &   33.09 &      87.46 & 45.48 & 12.25 & 1.75 & 3.08 & 7.42 & 1358.33 & 357.50 & 2926.92 & 18.00 & 28.25 & 27.64 & 80.47 \\
    \bottomrule
    \end{tabular}}
    \caption{Mean values for MSR = 8}
  \end{subtable}
 
 
  
  % table for msr=16
  \begin{subtable}[t]{\linewidth}
    \centering
    \resizebox{1.0\linewidth}{!}{
    \begin{tabular}{rrrrrrrrrrrrrrrrrrrrrr}
    \toprule
     QP &  MSR &    IDTP &   IDFP &    IDFN &  IDF1 &   IDP &   IDR &  Recall &  Precision &    F1 &    GT &   MT &   PT &   ML &      TP &     FP &      FN &   IDs &    FM &  MOTA &  MOTP \\
    \midrule
     18 &   16 & 1749.25 & 964.25 & 2536.00 & 50.48 & 61.51 & 44.07 &   62.04 &      88.73 & 71.88 & 12.25 & 4.50 & 3.33 & 4.42 & 2248.00 & 461.50 & 2037.25 & 25.08 & 35.25 & 54.03 & 82.08 \\
     22 &   16 & 1790.08 & 932.17 & 2495.17 & 50.40 & 61.45 & 43.96 &   61.39 &      88.42 & 71.38 & 12.25 & 4.58 & 3.17 & 4.50 & 2242.17 & 476.08 & 2043.08 & 24.33 & 35.75 & 53.36 & 82.16 \\
     26 &   16 & 1728.75 & 886.25 & 2556.50 & 50.48 & 62.82 & 43.28 &   60.43 &      89.89 & 71.21 & 12.25 & 4.33 & 3.42 & 4.50 & 2190.50 & 420.50 & 2094.75 & 23.58 & 34.08 & 53.36 & 82.40 \\
     30 &   16 & 1728.33 & 954.08 & 2556.92 & 48.91 & 61.02 & 41.83 &   60.04 &      90.00 & 71.10 & 12.25 & 4.17 & 3.42 & 4.67 & 2225.42 & 453.00 & 2059.83 & 23.17 & 34.25 & 53.21 & 82.26 \\
     34 &   16 & 1604.67 & 906.00 & 2680.58 & 46.94 & 60.73 & 38.90 &   56.89 &      90.40 & 69.10 & 12.25 & 3.83 & 3.17 & 5.25 & 2087.08 & 419.58 & 2198.17 & 21.67 & 32.08 & 50.49 & 82.29 \\
     38 &   16 & 1496.58 & 791.58 & 2788.67 & 45.47 & 62.23 & 36.46 &   53.13 &      91.95 & 66.47 & 12.25 & 3.58 & 3.08 & 5.58 & 1934.58 & 349.58 & 2350.67 & 23.17 & 31.83 & 47.81 & 82.04 \\
     42 &   16 & 1379.33 & 786.17 & 2905.92 & 39.76 & 58.32 & 31.38 &   47.28 &      90.98 & 60.56 & 12.25 & 2.92 & 3.25 & 6.08 & 1809.33 & 352.17 & 2475.92 & 20.33 & 33.58 & 41.72 & 81.33 \\
     46 &   16 & 1078.17 & 613.00 & 3207.08 & 32.42 & 61.75 & 23.68 &   33.48 &      88.70 & 45.99 & 12.25 & 1.92 & 2.83 & 7.50 & 1368.58 & 318.58 & 2916.67 & 18.42 & 28.58 & 27.67 & 79.77 \\
    \bottomrule
    \end{tabular}}
    \caption{Mean values for MSR = 16}
  \end{subtable}
 
 
 
  % table for msr=32
  \begin{subtable}[t]{\linewidth}
    \centering
    \resizebox{1.0\linewidth}{!}{
    \begin{tabular}{rrrrrrrrrrrrrrrrrrrrrr}
    \toprule
     QP &  MSR &    IDTP &   IDFP &    IDFN &  IDF1 &   IDP &   IDR &  Recall &  Precision &    F1 &    GT &   MT &   PT &   ML &      TP &     FP &      FN &   IDs &    FM &  MOTA &  MOTP \\
    \midrule
     18 &   32 & 1768.58 & 939.58 & 2516.67 & 50.67 & 61.70 & 44.31 &   61.98 &      88.62 & 71.78 & 12.25 & 4.58 & 3.25 & 4.42 & 2245.33 & 458.83 & 2039.92 & 25.00 & 35.58 & 53.80 & 82.08 \\
     22 &   32 & 1791.42 & 920.08 & 2493.83 & 50.38 & 61.48 & 43.92 &   61.49 &      88.46 & 71.42 & 12.25 & 4.58 & 3.25 & 4.42 & 2254.00 & 453.50 & 2031.25 & 25.00 & 36.42 & 53.15 & 82.18 \\
     26 &   32 & 1724.42 & 939.08 & 2560.83 & 49.68 & 61.44 & 42.82 &   60.80 &      89.44 & 71.32 & 12.25 & 4.42 & 3.08 & 4.75 & 2204.33 & 455.17 & 2080.92 & 24.25 & 35.75 & 53.41 & 82.34 \\
     30 &   32 & 1697.08 & 950.92 & 2588.17 & 49.14 & 61.34 & 41.91 &   59.79 &      89.85 & 70.91 & 12.25 & 4.17 & 3.42 & 4.67 & 2196.08 & 447.92 & 2089.17 & 24.08 & 36.58 & 52.83 & 82.25 \\
     34 &   32 & 1628.42 & 865.50 & 2656.83 & 48.13 & 61.98 & 39.98 &   57.10 &      90.70 & 69.32 & 12.25 & 4.00 & 3.00 & 5.25 & 2096.75 & 393.17 & 2188.50 & 21.42 & 31.58 & 50.83 & 82.38 \\
     38 &   32 & 1512.50 & 778.75 & 2772.75 & 45.10 & 61.90 & 36.13 &   53.20 &      91.29 & 66.31 & 12.25 & 3.58 & 3.08 & 5.58 & 1934.67 & 352.58 & 2350.58 & 20.75 & 31.08 & 47.57 & 81.98 \\
     42 &   32 & 1367.25 & 779.58 & 2918.00 & 39.92 & 58.70 & 31.38 &   47.28 &      90.61 & 60.48 & 12.25 & 3.00 & 3.42 & 5.83 & 1808.25 & 334.58 & 2477.00 & 19.67 & 32.25 & 41.48 & 81.16 \\
     46 &   32 & 1018.67 & 693.75 & 3266.58 & 30.47 & 57.32 & 22.52 &   33.53 &      88.54 & 45.93 & 12.25 & 1.75 & 3.08 & 7.42 & 1363.75 & 344.67 & 2921.50 & 19.25 & 28.25 & 27.97 & 81.11 \\
    \bottomrule
    \end{tabular}}
    \caption{Mean values for MSR = 32}
  \end{subtable}
  
  
  % table for msr=64
  \begin{subtable}[t]{\linewidth}
    \centering
    \resizebox{1.0\linewidth}{!}{
    \begin{tabular}{rrrrrrrrrrrrrrrrrrrrrr}
    \toprule
     QP &  MSR &    IDTP &   IDFP &    IDFN &  IDF1 &   IDP &   IDR &  Recall &  Precision &    F1 &    GT &   MT &   PT &   ML &      TP &     FP &      FN &   IDs &    FM &  MOTA &  MOTP \\
    \midrule
     18 &   64 & 1756.00 & 964.33 & 2529.25 & 50.10 & 61.04 & 43.73 &   62.17 &      88.82 & 72.02 & 12.25 & 4.67 & 3.00 & 4.58 & 2252.83 & 463.50 & 2032.42 & 24.75 & 36.75 & 54.24 & 82.06 \\
     22 &   64 & 1768.75 & 944.75 & 2516.50 & 50.68 & 61.77 & 44.22 &   61.56 &      88.53 & 71.48 & 12.25 & 4.42 & 3.33 & 4.50 & 2244.83 & 464.67 & 2040.42 & 24.75 & 36.75 & 53.33 & 82.12 \\
     26 &   64 & 1722.25 & 906.50 & 2563.00 & 49.46 & 61.60 & 42.30 &   60.18 &      89.24 & 70.90 & 12.25 & 4.33 & 3.25 & 4.67 & 2184.42 & 440.33 & 2100.83 & 24.33 & 34.50 & 52.62 & 82.37 \\
     30 &   64 & 1731.25 & 949.00 & 2554.00 & 49.44 & 61.59 & 42.27 &   59.80 &      89.16 & 70.64 & 12.25 & 4.33 & 3.08 & 4.83 & 2214.75 & 461.50 & 2070.50 & 23.83 & 34.50 & 52.47 & 82.28 \\
     34 &   64 & 1615.58 & 880.42 & 2669.67 & 46.90 & 60.65 & 38.91 &   57.32 &      91.40 & 69.69 & 12.25 & 3.83 & 3.25 & 5.17 & 2102.25 & 389.75 & 2183.00 & 20.92 & 32.17 & 51.51 & 82.42 \\
     38 &   64 & 1508.92 & 765.42 & 2776.33 & 46.25 & 63.23 & 37.09 &   52.56 &      90.69 & 65.69 & 12.25 & 3.50 & 3.08 & 5.67 & 1917.75 & 352.58 & 2367.50 & 20.33 & 31.83 & 46.48 & 81.96 \\
     42 &   64 & 1404.50 & 744.42 & 2880.75 & 42.43 & 62.55 & 33.18 &   47.31 &      90.25 & 60.65 & 12.25 & 3.00 & 3.33 & 5.92 & 1805.17 & 339.75 & 2480.08 & 18.83 & 32.75 & 41.57 & 81.38 \\
     46 &   64 &  995.00 & 729.42 & 3290.25 & 29.98 & 56.74 & 22.02 &   33.18 &      87.92 & 45.43 & 12.25 & 1.75 & 3.08 & 7.42 & 1364.33 & 356.08 & 2920.92 & 19.25 & 29.17 & 27.29 & 80.75 \\
    \bottomrule
    \end{tabular}}
    \caption{Mean values for MSR = 64}
  \end{subtable}

  
\label{tab:averaged_result_all}  
\end{table}
As discussed in Section \ref{sec:background/section_e}, the higher the QP, the lower the bitrate, so we expected the tracking performance to be lower. Our performance results from most metrics are consistent with this expectation. To explain the decrease of MOTA defined in terms of FP, FN, and IDs based on Equation \eqref{eqn:MOTA}, we can see that as QP increases, FP and IDs decreases but FN increases, which is consistent with our expectation. Although the decrease of FP and IDs contribute to the increase of MOTA, the increase of FN is significantly larger than FP and IDs; thus, MOTA decreases. However, we observed that Precision and MOTP are not entirely consistent with our expectation and that the performance first increases and then decreases as QP is increased. Since video samples differ in resolution, frame rate, number of objects, and object classes, tracking performance is different. Due to the relatively small number of video samples, the standard deviation is relatively high around the average performance. We show standard deviations in Appendix \ref{sec:appendix/section_std_dev}. To see if MSR impacts the performance scores, we visualized each performance score versus MSR as a horizontal axis, and we observed no significant dependence on MSR as shown in Figure \ref{fig:averaged_result_all_multiplots_msr}.
\begin{figure}[!htbp]
  \centering
  \includegraphics[width=1.0\linewidth]{img/averaged_all_multiplots_msr.pdf}
  \caption[Visualization of the averaged performance results at different MSR of all video samples]
  {Visualization of the averaged performance results at different MSR of all video samples.}
  \label{fig:averaged_result_all_multiplots_msr}
\end{figure}

% ----------------------------------------------------------
% Regression Analysis
% ----------------------------------------------------------

\subsection{Regression Analysis}
\label{subsec:/results/section_a/regression_analysis}
To examine the impact of QP and MSR on tracking accuracy quantitatively, we conducted a regression analysis on the entire data (all the video sequences). As we focused on the analysis on the MOTA score, we quantified the result on MOTA over QP and MSR values. As can be seen from the visualization, the relationship between MOTA and MSR is constant, but MOTA does change with QP. In order to conduct linear regression analysis, we transformed QP so that the relationship between MOTA and $\text{QP}'$ is more linear (we call the transformed QP as $\text{QP}'$). The determination of the exact QP transformation was found by plotting various forms of transformation on the scatter plots on the individual video sequences by trial and error. The factor 52 was chosen in the denominator because since QP ranges from 0 to 51 in theory \cite{sullivan_overview_2012} and if choose 51 as a factor, QP at 51 will be undefined. To avoid this issue, we chose the factor 52. Figure \ref{fig:QP_transformation} shows the scatter plots of MOTA before and after the chosen QP transformation on the example sequence BasketballPass. Using the form of Equation \eqref{eqn:QP_transformation}, the scatter plot of MOTA at different $\text{QP}'$ shows a linear relationship, as shown in Figure \ref{fig:QP_transformation_after}. Similar outcomes were observed in other video sequences as well.
% \begin{figure}[!tb]
%   \centering
%   \includegraphics[width=1.0\linewidth]{img/QP_transformation.pdf}
%   \caption[Scatter plots before and after the QP transformation]
%   {Scatter plots before and after the QP transformation.}
%   \label{fig:QP_transformation}
% \end{figure}

\begin{figure}[!tb]
  \centering
  \begin{subfigure}[b]{.5\textwidth}
    \includegraphics[width=\textwidth]{img/QP_transformation_before.pdf}
    \caption{MOTA vs QP scatter plot}
    \label{fig:QP_transformation_before}
  \end{subfigure}%
  \begin{subfigure}[b]{.5\textwidth}
    \includegraphics[width=\textwidth]{img/QP_transformation_after.pdf}
    \caption{MOTA vs ($\text{QP}'$) scatter plot}
    \label{fig:QP_transformation_after}
  \end{subfigure}
  \caption[Scatter plots before and after the QP transformation on the BasketballPass sequence]{%
    Scatter plots before and after the QP transformation on the BasketballPass sequence.%
  }
  \label{fig:QP_transformation}
\end{figure}
\begin{equation}
\text{QP}' =  \frac{1}{\frac{\text{QP}}{52}-1}
\label{eqn:QP_transformation}
\end{equation}
However, using all the video sequences, which have different MOTA scores, the variance at each $\text{QP}'$ is high, and little insight can be derived from the high variance of data as shown in Figure \ref{fig:MOTA_transformation_before}. Therefore, we transformed MOTA such that we subtract each MOTA score at different $\text{QP}'$ from the uncompressed MOTA score, as represented below.
% \begin{figure}[!tb]
%   \centering
%   \includegraphics[width=1.0\linewidth]{img/MOTA_transformation.pdf}
%   \caption[Scatter plots before and after the MOTA transformation]
%   {Scatter plots before and after the MOTA transformation.}
%   \label{fig:MOTA_transformation}
% \end{figure}

% spacing between subfigures will not align the sub figures horizontanlly

\begin{figure}[!tb]
  \centering
  \begin{subfigure}{.5\linewidth}
    \centering
    \includegraphics[width=\linewidth]{img/MOTA_transformation_before.pdf}
    \caption{MOTA vs $\text{QP}'$ scatter plot}
    \label{fig:MOTA_transformation_before}
  \end{subfigure}%
  \begin{subfigure}{.5\linewidth}
    \centering
    \includegraphics[width=\linewidth]{img/MOTA_transformation_after.pdf}
    \caption{$\text{MOTA}'$ vs $\text{QP}'$ scatter plot }
    \label{fig:MOTA_transformation_after}
  \end{subfigure}
  \caption[Scatter plots before and after the MOTA transformation on all the video sequences]{%
    Scatter plots before and after the MOTA transformation across all the video sequences.%
  }
  \label{fig:MOTA_transformation}
\end{figure}

\begin{equation}
\text{MOTA}_i' = \text{MOTA(Uncompressed)} -  \text{MOTA}(\text{QP}_i)
\label{eqn:MOTA_transformation}
\end{equation}
Doing this transformation, we only consider the drop of the MOTA score from the uncompressed sequence, and the variance is lower at higher $\text{QP}'$, in other words, the variance is lower at lower QP. This is expected because $\text{MOTA(Uncompressed)}$ and $\text{MOTA}(\text{QP}_i)$ are both random variables, and the difference of these two variables will be lower at lower QP. Figure \ref{fig:MOTA_transformation_after} shows the scatter plot of all the video sequences after the MOTA transformation. We can see from this figure that the relationship between $\text{MOTA}'$ and $\text{QP}'$ is still linear but error variance at each $\text{QP}'$ is not constant. Therefore, the weighted least squares method is adopted in the regression model. Also, since we have two continuous independent variables of $\text{QP}'$ and MSR and each may impact the continuous response variable of the $\text{MOTA}'$ score, we applied a multiple linear regression across all the video sequences. We represent the regression model as,
\begin{equation}
\text{MOTA}'_i = \beta_0 + \beta_1 \cdot \text{QP}'_i + \beta_2 \cdot \text{MSR}_i + \beta_3 \cdot \text{QP}'_i \cdot \text{MSR}_i + \epsilon_i~,
\label{eqn:regression_model}
\end{equation}
where $\beta_0$ is the intercept; $\beta_1$ and $\beta_2$ are the parameters of independent variables $\text{QP}'$ and MSR; $\beta_3$ is the parameter of the interaction term of $\text{QP}'$ and MSR. The interaction term is included to see if MSR and $\text{QP}'$ depend on each other. We assume that $\epsilon_i$ is the independent normally distributed error with mean 0 and variance $\sigma^2_i$, where $i$ indicates the $i$-th trial from the independent variables. As we are applying the weighted least squares method, the weight at each $\text{QP}'$ can be calculated as
\begin{equation}
    w_i = \frac{1}{\sigma^2_i}
    \label{eqn:weight}
\end{equation}
Equation \eqref{eqn:weight} is according to \cite[p.~422]{kutner_applied_2005}. Since $\sigma_i$ is the true value of standard deviation at each $\text{QP}'$, we estimate $w_i$ using the sample standard deviation $s_i$ of data points at each $\text{QP}'$:
\begin{equation}
    w_i = \frac{1}{s^2_i}
    \label{eqn:estimated_weight}
\end{equation}

%\setcounter{footnote}{1} % manully changing the footnote number. Referenced: https://tex.stackexchange.com/questions/359702/how-to-manually-reset-the-footnote-numbering-in-context
We utilized the Python package \cite{seabold_statsmodels_2010} for computation, and Table \ref{tab:regression} shows the results from the multiple linear regression analysis for the MOTA score.
\begin{table}[!tb]
    \centering
    \caption[Regression analysis result of the MOTA score for "all" object classes across all video sequences]
    {Regression analysis result of the MOTA score for "all" object classes across all video sequences.}
    \resizebox{0.25\linewidth}{!}{
\begin{tabular}{|c|c|}
\hline
$\beta_0$          & -4.32 \\
\hline
$\beta_1$          & -2.97 \\
\hline
$\beta_2$          & $< |10^{-2}|$ \\
\hline
$\beta_3$          & $< |10^{-2}|$ \\
\hline
p-value($\beta_0$) &  $< |10^{-6}|$ \\
\hline
p-value($\beta_1$) &  $< |10^{-9}|$ \\
\hline
p-value($\beta_2$) &  0.83 \\
\hline
p-value($\beta_3$) &  0.77 \\
\hline
\end{tabular}
    }
    \label{tab:regression}
\end{table}
To see if the impact of the independent variable is statistically significant on the dependent variable, we conducted a hypothesis test at a significance level of 0.05. The null hypothesis is the case the parameter is 0 and the alternative hypothesis is the case the parameter is not zero.
\begin{equation}
    \begin{aligned}
        H_0: \beta_c = 0 \\
        H_1: \beta_c \neq 0 \\
    \end{aligned}
\end{equation}
where $c=0,1,2,3$. From the table, the p-value for the test on $\beta_1$ is less than 0.05 while the p-values for the tests on $\beta_2$ and $\beta_3$ are greater than 0.05. This shows that we reject the null hypothesis of the test on $\beta_1$, so QP significantly impacts MOTA at 95\% confidence. However, we fail to reject the null hypothesis for $\beta_2$ and $\beta_3$. Hence, the available data is insufficient to prove that MSR and the interaction term have statistically significant impact on the MOTA score.

To further justify these results, we tested the adequacy of our regression model with the normal probability plot of the residuals. To plot this, we would have to obtain the residuals and the expected values of the residuals under normality. Since the normal probability plot requires variance to be constant and we have a model with unequal variance, we studentize the residuals in order to obtain the residuals with constant variance of 1. The studentized residuals $r_i$ can be obtained as
\begin{equation}
    r_i = \frac{ \text{MOTA}'_{i,\text{measurement}} - \text{MOTA}'_{i,\text{predicted}} }{s_i}
    \label{eqn:residuals}
\end{equation}
where $s_i$ is an estimator to the standard deviation $\sigma_i$. These studentized residuals will be sorted and be plotted in the normal probability plot. Adapted from \cite[p.~111]{kutner_applied_2005}, the expected values of studentized residuals under normality can be represented as the following equation.
\begin{equation}
    \text{E}(r_k) = \text{ppf}( \frac{k - 0.375}{n + 0.25} )
\end{equation}
ppf is a percent point function (ppf) and $k$ indicates the $k$-th smallest studentized residual. The range of $k$ is 1, 2, 3 ... $n$, and the smallest residual will be $k = 1$ and the largest residual will be $k = n$. This equation represents the sorted expected values of studentized residuals under the normal distribution with mean of 0 and variance of 1. Figure \ref{fig:normal_probability_plot} shows the normal probability plot of sorted studentized residuals and the sorted expected values of studentized residuals under normality.
\begin{figure}[!tb]
  \centering
  \includegraphics[width=0.8\linewidth]{img/normal_probability_plot.pdf}
  \caption[Normal probability plot to test the adequacy of the regression model]
  {Normal probability plot to test the adequacy of the regression model.}
  \label{fig:normal_probability_plot}
\end{figure}

Our samples are close to the normality which would be a straight line with a slope of 1.0. Fitting this plot, we obtain a slope of 1.001, intercept of 0.004, and the coefficient of determination $R^2$ of 0.985. The relatively high value of $R^2$ indicates that the error term $\epsilon_i$ is close to the normal distribution. Therefore, we conclude that our regression model \eqref{eqn:regression_model} is adequate. As we justified our regression model and from the hypothesis testing, we conclude that MSR does not impact the MOTA score, and QP and MSR do not depend on each other; therefore, our prediction for MSR, as explained in Section \ref{sec:background/section_e}, is inconsistent with our results. Thus, we will limit our further study to QP only.

% show the formulation
We showed that our regression model \eqref{eqn:regression_model} is appropriate and therefore, we propose to formulate the relationship between MOTA and QP. From this regression, we have the following formulations.
\begin{equation}
    \text{MOTA}' = \beta_0 + \beta_1 \cdot \text{QP}'
\end{equation}
\begin{equation}
    \text{MOTA} = \text{MOTA(Uncompressed)} - \beta_0 - \beta_1 \cdot \frac{1}{ \frac{\text{QP}}{52} - 1 }
\label{eqn:MOTA_vs_QP_formula}
\end{equation}
Substituting the estimated values of parameters from Table \ref{tab:regression}, we have the following equation.
\begin{equation}
    \text{MOTA} = 58.81 + 2.97 \cdot \frac{1}{ \frac{\text{QP}}{52} - 1 }
\label{eqn:MOTA_vs_QP_formula_substituted}
\end{equation}
We plotted the average values of MOTA measurements at each QP and overlaid the fitted model \ref{eqn:MOTA_vs_QP_formula_substituted} on the scatter plot as shown in Figure \ref{fig:MOTA_vs_QP_formulation}.
\begin{figure}[!tb]
  \centering
  \includegraphics[width=0.8\linewidth]{img/MOTA_vs_QP_formulation.pdf}
  \caption[Scatter plot of average values of MOTA measurements and fitted model]
  {Scatter plot of average values of MOTA measurements and fitted model.}
  \label{fig:MOTA_vs_QP_formulation}
\end{figure}

Note that our model is fitted according to the raw data points for "all" object classes while average MOTA values are plotted for visualizing purposes. Also, our model is fitted based on the QP range from 18 to 46, and since QP ranges from 0 to 51 in theory \cite{sullivan_overview_2012}, our model is extrapolated. However, from the practical point of view, QP = 51 is virtually never used and the QP range from 18 to 46 is sufficient. Future study could include the whole range of QP to fit the model. In addition, not only MOTA, the same analysis procedure can be applied to all tracking metrics, and we tabulated the regression analysis results of all the metrics in Appendix \ref{sec:appendix/section_regression_all}.

% ----------------------------------------------------------
% One-sided t-test
% ----------------------------------------------------------

\subsection{One-sided t-test}
\label{subsec:/results/section_a/t_test}
Finally, to quantify the results to answer the question - at which QP, the performance score is significantly lower than the score at the uncompressed case, we conducted the one-sided t-test. As explained in Section \ref{subsec:/results/section_a/regression_analysis}, without the MOTA transformation, the variance is high. To gain the better insight with the reduced variance, we consider the difference between the score at the uncompressed case and the score at each QP. For the one-sided t-test, we included all the metrics, so we will transform each performance score as
\begin{equation}
  p_{\text{diff}} =
    \begin{cases}
         p_{\text{comp}} - p_{\text{uncomp}} ,& \text{for IDFN, FN, PT, ML} \\        
         p_{\text{uncomp}} - p_{\text{comp}} ,& \text{else} \\
    \end{cases}
\end{equation}
where $p_{\text{comp}}$ is the performance score at each QP, $p_{\text{uncomp}}$ is the the performance score at the uncompressed case, and $p_{\text{diff}}$ represents the difference of such scores. Since the scores IDFN, FN, PT, and ML increase as QP increases according to Figure \ref{fig:averaged_result_all_multiplots_qp}, we subtract $p_{\text{comp}}$ from $p_{\text{uncomp}}$. For other metrics, the scores decrease as QP increases, so we subtract $p_{\text{uncomp}}$ from $p_{\text{comp}}$. Figure \ref{fig:score_transformation} shows that after the transformation of the score, MOTA as an example, the variance is reduced at lower QP.
\begin{figure}[!tb]
  \centering
  \begin{subfigure}{.5\linewidth}
    \centering
    \includegraphics[width=\linewidth]{img/score_transformation_before.pdf}
    \caption{MOTA vs QP scatter plot}
    \label{fig:score_transformation_before}
  \end{subfigure}%
  \begin{subfigure}{.5\linewidth}
    \centering
    \includegraphics[width=\linewidth]{img/score_transformation_after.pdf}
    \caption{$\text{MOTA}'$ vs QP scatter plot }
    \label{fig:score_transformation_after}
  \end{subfigure}
  \caption[Scatter plots before and after the MOTA transformation for t-test]{%
    Scatter plots before and after the MOTA transformation for t-test.%
  }
  \label{fig:score_transformation}
\end{figure}

To perform t-test, the hypotheses are
\begin{equation}
    \begin{aligned}
        H_0 : \mu_{\text{diff}} \leq 0 \\
        H_1 : \mu_{\text{diff}} > 0 \\
    \end{aligned}
\end{equation}
where $\mu_{\text{diff}}$ is the mean of 12 transformed score values $p_{\text{diff}}$. The null hypothesis is the case when the mean $\mu_{\text{diff}}$ is less than or equal to the population mean of 0. The alternative hypothesis is the case when the mean $\mu_{\text{diff}}$ is greater than the population mean of 0. Applying the one-sided t-test for "all" object classes across all the video sequences at a significance level of 0.05, we obtain the results in Table \ref{tab:t_test_all} using the Python package \cite{virtanen_scipy_2020}. The value less than the significance level of 0.05 is bolded in the table.
\begin{table}[!tb]
    \centering
    \caption[One-sided t-test for "all" object classes across all the video sequences]
    {One-sided t-test for "all" object classes across all the video sequences.}
    \resizebox{1.0\linewidth}{!}{
    
% \begin{tabular}{lllllllllllllllllllll}
\begin{tabular}{ccccccccccccccccccccc}
\toprule
{} &                 IDTP &                 IDFP &                 IDFN &                 IDF1 &            IDP &                  IDR &                Recall &      Precision &                    F1 &                    MT &    PT &                   ML &                    TP &                   FP &                    FN &                  IDs &                   FM &               mAP-50 &                  MOTA &                 MOTP \\
\midrule
p-value(QP=18) &                 0.44 &                 0.90 &                 0.44 &                 0.87 &           0.70 &                 0.90 &                  0.31 &  \textbf{0.02} &                  0.06 &         \textbf{0.05} &  0.79 &        \textbf{0.02} &                  0.29 &                 0.99 &                  0.29 &                 0.98 &                 0.94 &                 0.68 &         \textbf{0.03} &                 0.97 \\
p-value(QP=22) &                 0.91 &                 0.49 &                 0.91 &                 0.79 &           0.71 &                 0.79 &   $\pmb{< |10^{-3}|}$ &  \textbf{0.03} &   $\pmb{< |10^{-2}|}$ &         \textbf{0.02} &  0.63 &        \textbf{0.03} &                  0.39 &                 0.99 &                  0.39 &                 0.92 &                 0.93 &                 0.38 &   $\pmb{< |10^{-2}|}$ &                 0.97 \\
p-value(QP=26) &                 0.16 &                 0.07 &                 0.16 &                 0.58 &           0.91 &                 0.20 &   $\pmb{< |10^{-5}|}$ &           0.89 &   $\pmb{< |10^{-3}|}$ &   $\pmb{< |10^{-3}|}$ &  0.59 &  $\pmb{< |10^{-3}|}$ &   $\pmb{< |10^{-2}|}$ &                 0.27 &   $\pmb{< |10^{-2}|}$ &                 0.53 &                 0.41 &        \textbf{0.02} &   $\pmb{< |10^{-3}|}$ &                 1.00 \\
p-value(QP=30) &                 0.08 &                 0.76 &                 0.08 &                 0.24 &           0.65 &        \textbf{0.03} &   $\pmb{< |10^{-5}|}$ &           0.83 &   $\pmb{< |10^{-2}|}$ &   $\pmb{< |10^{-3}|}$ &  0.50 &  $\pmb{< |10^{-2}|}$ &                  0.10 &                 0.84 &                  0.10 &                 0.29 &                 0.47 &                 0.08 &   $\pmb{< |10^{-3}|}$ &                 0.99 \\
p-value(QP=34) &  $\pmb{< |10^{-2}|}$ &  $\pmb{< |10^{-2}|}$ &  $\pmb{< |10^{-2}|}$ &  $\pmb{< |10^{-2}|}$ &           0.60 &  $\pmb{< |10^{-4}|}$ &   $\pmb{< |10^{-6}|}$ &           1.00 &   $\pmb{< |10^{-4}|}$ &   $\pmb{< |10^{-5}|}$ &  0.95 &  $\pmb{< |10^{-5}|}$ &   $\pmb{< |10^{-3}|}$ &        \textbf{0.01} &   $\pmb{< |10^{-3}|}$ &  $\pmb{< |10^{-3}|}$ &  $\pmb{< |10^{-2}|}$ &        \textbf{0.01} &   $\pmb{< |10^{-4}|}$ &                 0.99 \\
p-value(QP=38) &  $\pmb{< |10^{-3}|}$ &  $\pmb{< |10^{-6}|}$ &  $\pmb{< |10^{-3}|}$ &  $\pmb{< |10^{-4}|}$ &           0.88 &  $\pmb{< |10^{-6}|}$ &   $\pmb{< |10^{-7}|}$ &           1.00 &   $\pmb{< |10^{-6}|}$ &   $\pmb{< |10^{-7}|}$ &  0.92 &  $\pmb{< |10^{-7}|}$ &   $\pmb{< |10^{-4}|}$ &  $\pmb{< |10^{-4}|}$ &   $\pmb{< |10^{-4}|}$ &  $\pmb{< |10^{-2}|}$ &  $\pmb{< |10^{-2}|}$ &  $\pmb{< |10^{-3}|}$ &   $\pmb{< |10^{-5}|}$ &                 0.57 \\
p-value(QP=42) &  $\pmb{< |10^{-5}|}$ &  $\pmb{< |10^{-4}|}$ &  $\pmb{< |10^{-5}|}$ &  $\pmb{< |10^{-6}|}$ &           0.17 &  $\pmb{< |10^{-7}|}$ &  $\pmb{< |10^{-10}|}$ &           1.00 &   $\pmb{< |10^{-8}|}$ &   $\pmb{< |10^{-9}|}$ &  0.40 &  $\pmb{< |10^{-6}|}$ &   $\pmb{< |10^{-8}|}$ &  $\pmb{< |10^{-5}|}$ &   $\pmb{< |10^{-8}|}$ &  $\pmb{< |10^{-3}|}$ &                 0.12 &  $\pmb{< |10^{-5}|}$ &   $\pmb{< |10^{-8}|}$ &        \textbf{0.01} \\
p-value(QP=46) &  $\pmb{< |10^{-8}|}$ &  $\pmb{< |10^{-4}|}$ &  $\pmb{< |10^{-8}|}$ &  $\pmb{< |10^{-9}|}$ &  \textbf{0.04} &  $\pmb{< |10^{-9}|}$ &  $\pmb{< |10^{-13}|}$ &           0.23 &  $\pmb{< |10^{-12}|}$ &  $\pmb{< |10^{-13}|}$ &  0.69 &  $\pmb{< |10^{-8}|}$ &  $\pmb{< |10^{-11}|}$ &        \textbf{0.02} &  $\pmb{< |10^{-11}|}$ &  $\pmb{< |10^{-2}|}$ &        \textbf{0.01} &  $\pmb{< |10^{-9}|}$ &  $\pmb{< |10^{-11}|}$ &  $\pmb{< |10^{-3}|}$ \\
\bottomrule
\end{tabular}

    }
\label{tab:t_test_all}
\end{table}
From this result, for example, the MOTA score at QP = 18 is significantly lower than the score of the uncompressed sequence with 95\% confidence as we reject the null hypothesis. In fact, we can conclude for most of the metrics except Precision and PT as following; the performance score is greater than the uncompressed score for the metrics IDFN, FN, and ML but the performance score is less than the uncompressed score for other metrics at specific QP with 95\% confidence. From Figure \ref{fig:averaged_result_all_multiplots_qp}, the Precision score increases first and then decreases as QP increases. Therefore, the result from the one-sided t-test is consistent with the the visualization. According to Equation \ref{eqn:Precision}, Precision depends on TP and FP, and the increase of p-value in FP can be confirmed at QP = 30. The detailed reasoning for this outcome is explained in the following Section \ref{sec:results/section_b}. A slight increase of MOTP can also be confirmed from QP = 26. We also applied the one-sided t-test to the "person" object class and similar conclusions can be made. See Appendix \ref{sec:appendix/section_t_test_0} for the "person" object case.

As we just showed the results from the visualization and the statistical analysis for the multiple video sequences, most metrics behave in a way as expected, but the outcomes of Precision and MOTP were not entirely expected. We will further analyze individual sequences in the next section.


% As \cite{sharrab_modeling_2017} shows that bitrate is inversely proportional to QP and HEVC range from 0 to 51 \cite{sullivan_overview_2012}, our proposed formula 