\chapter{Conclusion}
\label{chap:conclusion}

In this work, we studied the effects of video compression on multiple object tracking performance. QP and MSR have been selected to examine the effects. We collected the data by running the video compression standard of HEVC to the uncompressed video sequences and run the multiple object tracker. Analyzing the data on both results from the compressed and uncompressed video sequences, we understood that MSR does not significantly impact the tracking performance. For QP, the averaged result of 12 video sequences shows that as QP increases, the performance metrics decrease, which is consistent with the hypothesis, except for Precision and MOTP. By analyzing the individual sequences as case studies, we proved that FN increases as QP increases due to less detection as the image quality drops. Since FN is significantly larger than the change in FP and IDs, we observed MOTA be decreased. However, We found that MOTA increased midway through the QP range in some sequences. This is because the problematic type of objects causes YOLOv3 to incorrectly detect the objects as other object classes, leading to a higher FP than the case in the uncompressed video sequence. As the image quality drops, the frequency of incorrect detection decreases. Hence, the performance increases, but as the image quality further drops, MOTA decreases due to the significant decrease of FN. The reason the Precision increased in the averaged result is also due to the decrease of FP. As we are analyzing the problem in the multiple objects, occlusion is inevitable in some video sequences. We found that IDs decreased in the video sequences with the frequent occlusions due to the less detected occlusions as QP increases. For the sequences with fewer occlusions, IDs increased because of discontinuous detection, causing identity switches, becoming more significant. For MOTP, an increase of performance midway through the QP range was observed. From the analysis of individual sequences, we observed such increases in some sequences, indicating that as the image quality drops, the object localization improves in some sequences, but further analysis is required to identify its cause.

We learned that all these results are due to the ability of YOLOv3 and SORT, and hence the tracking performance is highly dependent on the design of the tracker. In our case, we are using a detection-based tracker; therefore, the tracking performance is dependent on the detection performance. We employed the simplest possible approach to detecting and tracking multiple objects using YOLOv3 and SORT; however, since there are better detectors and trackers, we could run the experiment with a better multiple object tracker. For example, for the detector, Ultralytics recently developed YOLOv5x6, achieving mAP of 54.4 and AP\textsubscript{50} of 72.0 at a better resolution of 1280x1280 \cite{jocher_ultralyticsyolov5_2021}. This is a better performance than what we have in YOLOv3 as mAP of 43.3 and AP\textsubscript{50} of 63.0 at a resolution of 640x640. Since SORT does not have a feature to perform re-ID for the long-term occlusions and undetection, we could improve the tracker by implementing a re-ID feature. Finally, since QP and MSR are not the only configurations in the video compression standard of HEVC, we could explore and examine further with different configurations.