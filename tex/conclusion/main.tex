\chapter{Conclusion}
\label{chap:conclusion}

In this work, we studied the impact of video compression on multiple object tracking performance. We started with annotating the ground truth that is suitable for object tracking, conducted the experiment by running the HEVC codec at different QP and MSR to the uncompressed video sequences, and ran the multiple object tracker to the decoded sequences. Analyzing the results from the compressed and uncompressed video sequences, the visualization of average results across 12 video sequences shows that as QP increases, detection and tracking performance decrease, which is consistent with the expectation, except for the Precision and MOTP scores. Applying the regression analysis to all the video sequences, we learned that MSR does not impact the MOTA score but QP does with 95\% confidence based on the available data. We also proposed the formulation of MOTA and QP. One-sided t-test was also conducted to determine the specific QP at which detection and tracking performance on compressed sequences is lower than on uncompressed sequences. 

By analyzing further on four individual sequences as case studies, we illustrated that FN increases as QP increases due to less detection as the image quality drops. Since FN is significantly larger than the change in FP and IDs, we observed a decrease of MOTA. However, We found that MOTA sometimes increases midway through the QP range in some sequences. This is because the certain type of objects causes YOLOv3 to detect wrong objects, leading to a higher FP than the case on uncompressed video sequences. As the image quality drops, the frequency of incorrect detection decreases; hence, FP decreases and MOTA increases. However, as the image quality further drops, MOTA decreases due to the significant increase of FN. The reason the Precision score increased in the average results is also due to the decrease of FP. As we are analyzing the problem in the multiple objects, occlusion is inevitable in some video sequences. We found that IDs decreases in the video sequences with frequent occlusions due to the less detected occlusions as QP increases. For sequences with fewer occlusions, IDs increases as QP increases due to "gap" of trajectories caused by the drop of image quality. For MOTP, the increase of score midway through the QP range was observed. From the analysis of four individual sequences, we observed such increases in some sequences, which indicate that as the image quality drops, the object localization improves in some sequences, but further analysis is required to identify its cause. 

These case studies explained that the results are due to the design of YOLOv3 and SORT. Tracking performance is dependent on the design of the tracker. In our case, we are using a detection-based tracker; therefore, the tracking performance is dependent on the detection performance. Analyzing how detection impacts tracking performance, the linear and non-linear relationships of MOTA and mAP-50 were observed. The more thorough analysis will be necessary to identify the reason for the non-linear outcome, which is a decrease of MOTA growth rate as mAP-50 increases. The possible factors to consider could be size of bounding boxes and parameters in the Kalman filter framework in SORT.

We employed a simple and effective approach to detecting and tracking multiple objects using YOLOv3 and SORT; however, since there are better detectors and trackers, we could run the experiment with a better multiple object tracker. For example, for a detector, Ultralytics recently developed YOLOv5x6, achieving mAP of 54.4 and mAP-50 of 72.0 at a better resolution of $1280 \times 1280$ \cite{jocher_ultralyticsyolov5_2021}. This is a better performance than what we have in YOLOv3, i.e., mAP of 43.3 and mAP-50 of 63.0 at a resolution of $640 \times 640$. Since SORT does not have a feature to perform re-ID for the long-term occlusions and undetection, we could improve the tracker by implementing a re-ID feature. Finally, since QP and MSR are not the only configuration parameters in the video compression standard of HEVC, we could explore and examine further with different configurations.