\chapter{Introduction}
\label{chap:introduction}

Object detection is a computer vision task that can recognize the object instances from digital images. Specifically, the object detection task detects the different object classes such as "person", "cup", and "clock". Understanding object detection allows us to achieve various applications such as “smart vehicle”, the system that informs the driver if one is running in the right direction at a proper speed, or a system to detect pedestrians to prevent accidents utilizing real-time object detection \cite{gavrila_real-time_1999}. Object detection task can further be developed to object tracking, which is another computer vision task. Object detection task is able to detect the objects for different classes, but this task alone has no unique identification of objects within the same class in a sequence of image frames. For example, the system can detect multiple persons, but there is no identification of each person across the frames, and hence object tracking task will assign each unique ID to each object instance.

Video compression is ubiquitous in most visual processing pipelines. Compression in the video is necessary because it would be impractical to transmit the video and save it in the storage device without compression. For example, 720x480 pixels of full-color 90 min video with 30 frames per second (fps) has 167.96 GB from the group of \citeauthor{ponlatha_comparison_2013} \cite{ponlatha_comparison_2013} and is too huge in size to transmit. Since video compression is ubiquitous, any video we access is all pre-compressed, and this raises the question of how much video compression impacts the object tracking performance. To our knowledge, the effect of video compression on object tracking has not been studied in detail. Hence, this project aims to analyze the effect of video compression on object tracking performance by assessing the various metrics used for multiple object tracking.

\section{Object Detection and Tracking}
\label{sec:introduction/section_a}

Object tracking problem can be categorized into Single Object Tracking (SOT) and Multiple Object Tracking (MOT). SOT tracks a single object over the sequence of digital images, while MOT tracks multiple objects. The following introduction to MOT has been explained well by \citeauthor{luo_multiple_2021} \cite{luo_multiple_2021}. SOT tracks a single object with a focus on appearance and motion models. MOT has common challenges with the single object case, but tracking multiple objects involves additional challenges to be solved. In the multiple objects case, the tracking system needs to determine the number of objects that could vary over the frames and preserve their identities. In this thesis work, we do not want to restrict tracking to the single object case, but aim to analyze the effect of compression on tracking multiple objects, which is more general.

\citeauthor{luo_multiple_2021} \cite{luo_multiple_2021} also explained the general categories of MOT. MOT initialization methods can be grouped into Detection-Based Tracking and Detection-Free Tracking. Detection-Based Tracking involves an object detector to perform detection as initialization to the tracker for every image in a sequence. The object tracker will then assign a unique identity to each detected object, in other words, performing data association of unique ID based on detected objects. Detection-Free Tracking does not involve an object detector but requires other initialization methods such as inference that exploits rich appearance and motion information \cite{lin_detection-free_2016}. MOT processing methods can be group into online tracking and offline tracking. Online tracking performs data association of unique identities based on current and past observations, while offline tracking does so by future, current, and past observations as a batch. The type of output from MOT can also be stochastic or deterministic. The output of stochastic tracking varies every time the tracker is run while for the deterministic case, the output does not change over the tracker runs.

For this thesis, we are using Detection-Based Tracking, and hence we will need an object detector as part of tracking. Object detection has been actively researched for the past 20 years, and there are two stages the research has gone through according to \citeauthor{zou_object_2019} \cite{zou_object_2019}. The first stage is traditional detection, which involves hand-crafted methods while the second stage involves deep learning-based methods. Deep learning-based methods can be further broken down into two categories: one-stage detection and two-stage detection. For two-stage detectors, the first stage generates the region of interest as a bounding box detection proposal and each region proposal is feed into a convolutional neural network (CNN). Classification and regression is performed on each proposal in the second stage. Various two-stage deep learning-based methods have been developed such as Region Based Convolutional Neural Network (R-CNN) \cite{girshick_rich_2014}, Spatial Pyramid Pooling network (SPP-net) \cite{he_spatial_2015}, Fast R-CNN \cite{girshick_fast_2015}, Faster R-CNN \cite{ren_faster_2017}, Feature Pyramid Network (FPN) \cite{lin_feature_2017}, and Mask R-CNN \cite{he_mask_2017}. As for one-stage detectors, the full image is feed into the single neural network and each bounding box as a region of interest is predicted by the network. The examples for the one-stage detector could be You Only Look Once (YOLO) \cite{redmon_you_2016}, Single Shot Multibox Detector (SSD) \cite{liu_ssd_2016}, and RetinaNet \cite{lin_focal_2017}.

\citeauthor{luo_multiple_2021} \cite{luo_multiple_2021} listed the examples of components used in multiple object tracker: appearance model that computes the affinity between the observations based on the visual cues and statistical measures, motion model that captures dynamic behavior of objects such as linear and non-linear motion, interaction model that considers individual interaction in the social environment, exclusion model that utilizes constraint between objects or trajectories, occlusion handling, and finally inference approach that utilizes probability distribution of object states or approach based on the deterministic optimization.

Out of all aforementioned possible ways to detect and track objects, as listed above, we employed the YOLOv3 object detector \cite{redmon_yolov3_2018} with Simple Online Realtime Tracking (SORT) \cite{bewley_simple_2016}. This solution is detection-based and online, with deterministic output, and the details are explained in Chapter \ref{chap:background}.


\section{Video Compression}
\label{sec:introduction/section_b}

The early video compression form is first described by Ray Davis Kell in 1929 as the difficulty of transmitting the whole successive images of video can be avoided by only sending the difference between the successive images, though it was not actually used however became the foundation for the video compression standards today \cite{jacobs_brief_2009}. \citeauthor{jacobs_brief_2009} explained that the video system was originated from the oscilloscope with Cathode Ray Tubes. The early video compression was the analog system but the digital video processing has been developed and is widely used today. \citeauthor{zhang_overview_2019} explained the concept of typical video compression today as following \cite{zhang_overview_2019}. The video compression consists of the encoder to compress the images into the compressed form which can be stored or transmits to the other location, and decoder to de-compress the images. This process of coding and de-coding is also called codec. The typical video compression standards nowadays comprise of predictive coding, transform coding, and entropy coding as shown in Fig. \ref{fig:comp_architecture}. Predictive coding is the component that reduces the inter-frame temporal redundancy and intra-frame spatial redundancies of video by motion estimation (ME) and motion compensation (MC) techniques. Transform coding is the component where the transform coefficients that are quantized are generated through discrete cosine transforms (DCT) to help reducing the spatial dependencies. Entropy coding is the component where compressed bit streams are generated.

\begin{figure}[htb]
  \centering
  \includegraphics[width=0.8\linewidth]{img/comp_architecture.png}
  \caption[Typical Architecture of Video Compression]{
    
  }
  \label{fig:comp_architecture}
\end{figure}

There are two type of video coding; losssless coding and lossy coding. Lossless coding compresses the images and reconstructed images can be obtained after de-compression without any loss of information. Lossy coding however compresses the images by removing the less important information, which will sacrifice the image quality to the level the human visual system can be tolerant. Lossy compression is more widely used today since it allows much lower compressed size and more efficient than the lossless manner.

Since the first video compression standard of H.120 that has been developed in 1984, the various video compression standards have been developed such as MPEG and H.26X series \cite{zhang_overview_2019}. The organization of Moving Picture Experts Group (MPEG) in the International Standards Organization (ISO) and the International Electrotechnical Commission (IEC) are developing the MPEG series such as MPEG-1, MPEG-2, and MPEG-4.

The organization of Video Coding Expert Group (VCEG) of International Telecommunication Union (ITU-T) are developing H.26X series starting from the first standard of H.120, then developed H.261, H.262, H.263, H.264 (AVC), and H.265 (HEVC). H.264 or so called Advanced Video Coding (AVC) were developed in 2003 and the typical architecture shown in Fig.\ref{fig:comp_architecture} were followed since H.264/AVC. H.264/AVC is the most widely used standard nowadays and supports up to 4k resolution of video. H.265 or so called High Efficiency Video Coding (HEVC) was developed based on H.264/AVC structure and is a more recent standard that has been developed in 2013. H.265/HEVC supports the up to 8k resolution of the video but is not yet widely supported. There have been more recent development of standards such as, for example, Versatile Video Coding (VVC) and AOMedia Video 1 (AV1). These newly developed standards after the predecessor H.265/HEVC achieve better coding performance and will enable higher quality and efficiency, VR system, and 360-degree of video applications. However, these recently developed standards including H.265/HEVC are still yet to be supported due to the lack of current computational power by the hardware. As a possible future generation of video compression standards after VVC and AV1, there have been research in compression technologies that utilizes machine learning for video coding, hardware acceleration and parallel computing. Out of these video compression standards, we have adopted H.265/HEVC for the experiment in this thesis.

\section{Thesis preview}
\label{sec:introduction/thesis_preview}

As an organization of this thesis, we start with illustrating Chapter 1 by the motivation for this research, and provides history and background information for each concept of object tracking and video compression. We then explain the more detail background information on the adopted methods in Chapter 2. Chapter 3 shows the methodologies and experimental procedures. Chapter 4 illustrates the results from the experiment and data analysis. Finally, we will conclude this research with the highlighted insights in Chapter 5.
