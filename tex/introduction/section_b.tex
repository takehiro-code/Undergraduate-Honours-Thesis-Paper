\section{Video Compression}
\label{sec:introduction/section_b}

The early video compression form is first described by Ray Davis Kell in 1929 as the difficulty of transmitting the whole successive images of video can be avoided by only sending the difference between the successive images, though it was not actually used however became the foundation for the video compression standards today \cite{jacobs_brief_2009}. \citeauthor{jacobs_brief_2009} explained that the video system was originated from the oscilloscope with Cathode Ray Tubes. The early video compression in the analog system was interlacing which transmits the even and odd lines alternatively but the digital video processing has been developed and is widely used today. Video compression consists of the encoder to compress the images into the compressed form which can be stored or transmits to the other location, and decoder to de-compress the images. This coding and de-coding is also called codec. The typical video compression standards nowadays comprise of predictive coding, transform coding, and entropy coding. Predictive coding is the component that reduces the information
There are two type of video coding; losssless coding and lossy coding. Lossless coding compresses the images and reconstructed images can be obtained after de-compression without any loss of information. Lossy coding however compresses the images by removing the less important information, which will sacrifice the image quality to the level the human visual system can be tolerant. Lossy compression is more widely used today since it allows much lower compressed size and more efficient than the lossless manner.




Since then various video compression standards have been developed such as MPEG and H.26X series \cite{zhang_overview_2019} and is listed as following.

\begin{itemize}

\item \textbf{H.120}: The first digital video compression standard that has been developed in 1984 but is depreciated today.

\item \textbf{H.261}: 


\end{itemize}


