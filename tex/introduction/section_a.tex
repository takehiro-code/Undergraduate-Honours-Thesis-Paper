\section{Object Detection and Tracking}
\label{sec:introduction/section_a}

Object tracking problem can be categorized into Single Object Tracking (SOT) and Multiple Object Tracking (MOT). SOT tracks a single object over the sequence of digital images, while MOT tracks multiple objects. The following introduction to MOT has been explained well by \citeauthor{luo_multiple_2021} \cite{luo_multiple_2021}. SOT tracks a single object with a focus on appearance and motion models. MOT has common challenges with the single object case, but tracking multiple objects involves additional challenges to be solved. In the multiple objects case, the tracking system needs to determine the number of objects that could vary over the frames and preserve their identities. In this thesis work, we do not want to restrict tracking to the single object case, but aim to analyze the effect of compression on tracking multiple objects, which is more general.

\citeauthor{luo_multiple_2021} \cite{luo_multiple_2021} also explained the general categories of MOT. MOT initialization methods can be grouped into Detection-Based Tracking and Detection-Free Tracking. Detection-Based Tracking involves an object detector to perform detection as initialization to the tracker for every image in a sequence. The object tracker will then assign a unique identity to each detected object, in other words, performing data association of unique ID based on detected objects. Detection-Free Tracking does not involve an object detector but requires other initialization methods such as inference that exploits rich appearance and motion information \cite{lin_detection-free_2016}. MOT processing methods can be group into online tracking and offline tracking. Online tracking performs data association of unique identities based on current and past observations, while offline tracking does so by future, current, and past observations as a batch. The type of output from MOT can also be stochastic or deterministic. The output of stochastic tracking varies every time the tracker is run while for the deterministic case, the output does not change over the tracker runs.

For this thesis, we are using Detection-Based Tracking, and hence we will need an object detector as part of tracking. Object detection has been actively researched for the past 20 years, and there are two stages the research has gone through according to \citeauthor{zou_object_2019} \cite{zou_object_2019}. The first stage is traditional detection, which involves hand-crafted methods while the second stage involves deep learning-based methods. Deep learning-based methods can be further broken down into two categories: one-stage detection and two-stage detection. For two-stage detectors, the first stage generates the region of interest as a bounding box detection proposal and each region proposal is feed into a convolutional neural network (CNN). Classification and regression is performed on each proposal in the second stage. Various two-stage deep learning-based methods have been developed such as Region Based Convolutional Neural Network (R-CNN) \cite{girshick_rich_2014}, Spatial Pyramid Pooling network (SPP-net) \cite{he_spatial_2015}, Fast R-CNN \cite{girshick_fast_2015}, Faster R-CNN \cite{ren_faster_2017}, Feature Pyramid Network (FPN) \cite{lin_feature_2017}, and Mask R-CNN \cite{he_mask_2017}. As for one-stage detectors, the full image is feed into the single neural network and each bounding box as a region of interest is predicted by the network. The examples for the one-stage detector could be You Only Look Once (YOLO) \cite{redmon_you_2016}, Single Shot Multibox Detector (SSD) \cite{liu_ssd_2016}, and RetinaNet \cite{lin_focal_2017}.

\citeauthor{luo_multiple_2021} \cite{luo_multiple_2021} listed the examples of components used in multiple object tracker: appearance model that computes the affinity between the observations based on the visual cues and statistical measures, motion model that captures dynamic behavior of objects such as linear and non-linear motion, interaction model that considers individual interaction in the social environment, exclusion model that utilizes constraint between objects or trajectories, occlusion handling, and finally inference approach that utilizes probability distribution of object states or approach based on the deterministic optimization.

Out of all aforementioned possible ways to detect and track objects, as listed above, we employed the YOLOv3 object detector \cite{redmon_yolov3_2018} with Simple Online Realtime Tracking (SORT) \cite{bewley_simple_2016}. This solution is detection-based and online, with deterministic output, and the details are explained in Chapter \ref{chap:background}.
