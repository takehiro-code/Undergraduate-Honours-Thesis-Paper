\section{High Efficiency Video Coding (HEVC)}
\label{sec:background/section_c}

H.265/HEVC is the video compression standard developed by the ITU-T Video Coding Expert Group in 2013. H.265/HEVC follows the same structure of its predecessor H.264/AVC, as shown in Figure \ref{fig:comp_architecture} but achieves a better coding performance \cite{zhang_overview_2019}. \citeauthor{zhang_overview_2019} \cite{zhang_overview_2019} explains that the typical structure consists of predictive coding with intra-frame and inter-frame prediction, transform coding that generates the quantized transform coefficients from DCT, and entropy coding that generates a compressed bitstream. To obtain a de-compressed video, HEVC performs entropy decoding, transform decoding, and predictive decoding, in a reversed order to the encoding part. For the experiments, we have used HEVC test model (HM) version 16.20, and varied two compression parameters: quantization parameter (QP) and motion search range (MSR).

\subsection{Quantization Parameter}
\label{subsec:background/section_c/qp}
Quantization parameter (QP) is a parameter used in transform coding. The value of QP determines the quantization step size by which we obtain the quantized transform matrix. QP ranges from 0 to 51, and an increase of 6 in QP will double the quantization step size \cite{sullivan_overview_2012} \cite{budagavi_hevc_2014}. According to \cite{sharrab_modeling_2017}, QP has a significant impact on the bitrate. They showed that bitrate is inversely proportional to QP and is linearly proportional to the pixel rate. Note that they refer pixel rate as "frame rate multiplied by number of pixels in the frame" \cite[p.~7]{sharrab_modeling_2017}. This means that high QP will result in lower bitrate and hence the lower amplitude resolution. In other words, a high QP will cause amplitude resolution to be lower, while a low QP will sustain a high amplitude resolution.

\subsection{Motion Search Range}
\label{subsec:background/section_c/msr}
Motion estimation is an inter-frame prediction technique that finds the best match for a block of pixels between the reference frame and the current frame while minimizing the rate-distortion cost or maximizing correlation. Motion estimation reduces temporal redundancy by obtaining a motion vector that points from the target candidate region in the current frame toward the region in the previous reference frame. The region where the block matching is performed is the search window, and its size is called the search range (SR). A high SR value means a larger search window and hence requires more memory and computation, but a low SR means a smaller search window and requires less memory \cite{lou_adaptive_2010} \cite{bachu_review_2015}. From this logic, we can interpret that large SR could cover fast motion in a video, while the low SR only covers the slower motion. In the experiment, we call this parameter motion search range (MSR), and we examine its impact on tracking accuracy.