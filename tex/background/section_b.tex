\section{Simple Online Realtime Tracking (SORT)}
\label{sec:background/section_b}

The detection-based tracking requires an object detector to initialize the state by detecting the objects, and a tracking component will be required. Multiple object tracking problem can be viewed as a data association problem in assigning the optimal unique identity to each object. Simple Online Realtime Tracking (SORT) was considered for the tracking component, which is detection-based and online tracker with a deterministic output. As \citeauthor{bewley_simple_2016} describes \cite{bewley_simple_2016}, based on the detected objects, SORT utilizes the constant linear motion model to estimate the displacements of the objects across the frames. The model can be represented in the following equation,
\begin{equation}
\textbf{x} = [u, v, s, r, \dot{u}, \dot{v}, \dot{s}]^T,
\label{eq:SORT_model}
\end{equation}
where $u$ and $v$ are the pixel coordinates in the horizontal and vertical direction; $s$ and $r$ are the area and aspect ratio of the bounding box; $\dot{u}$ and $\dot{v}$ represent the velocity; $\dot{s}$ is the rate of change of the area, which can be solved optimally by the Kalman filter framework. As for the data association, the assignment cost matrix, consisting of IOU values between each detected object, can be computed. The assignments of unique identities can be solved optimally from this matrix via the Hungarian algorithm. When the detected objects overlap less than the minimum IOU value, untracked objects are found, and followed by the minimum number of detections, the trajectories with the unique identities are initialized. During the initialization, the velocities are set to 0. When the objects are not detected for a certain number of frames labeled as $T\textsubscript{Lost}$, then the trajectories are terminated.

SORT has been focused on simplicity of design, which could serve as a baseline method. This is because the design of SORT is essentially consisting of a Kalman filter framework for predicting motion and a Hungarian algorithm for data association; however, it does not deal with the long-term occlusion. Adding the object re-identification feature to deal with occlusion will weigh significant complexities on the tracker. SORT is simple yet fast and maintains high accuracy with deterministic output. Combined with a fast YOLOv3 object detector and SORT, it will be easy for us to run the experiment since both components are fast enough and produces the same result every time we run the same experiment.