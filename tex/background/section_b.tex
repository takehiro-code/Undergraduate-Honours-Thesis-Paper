\section{Simple Online Realtime Tracking (SORT)}
\label{sec:background/section_b}

The detection-based tracking requires object detector to initialize the state by detecting the objects and tracking component will be required. Multiple object tracking problem can be viewed as a data association problem in assigning the optimal unique identities to each object. Simple Online Realtime Tracking (SORT) was considered for the tracking component, which is detection-based and online tracker with the deterministic output. As \citeauthor{bewley_simple_2016} describes \cite{bewley_simple_2016}, based on the detected objects, SORT utilizes the constant linear motion model to estimate the displacements of the objects across the frames. The model can be represented in the following equation as,
\begin{equation}
\textbf{x} = [u, v, s, r, \dot{u}, \dot{v}, \dot{s}]^T
\label{eq:SORT_model}
\end{equation}
where $u$ and $v$ are the pixel coordinates in horizontal and vertical direction, $s$ and $r$ are the area and aspect ratio of the bounding box. $\dot{u}$ and $\dot{v}$ represents the velocity and $s$ is the rate of change of the area, which can be solved optimally by the Kalman filter framework. As for the data association, the assignment cost matrix consisting of IOU values between each detected object can be computed, and the assignments of unique identities can be solved optimally from this matrix via Hungarian Algorithm. When the detected objects overlaps less than the minimum IOU value, untracked objects are found the trajectories with the unique identities are initialized. During the initialization, the velocities are set to 0. When the objects are not detected for a certain number of frames labeled as $T\textsubscript{Lost}$, then the trajectories are terminated.

SORT has been focused in simplicity of design, which could serve as a baseline method. This is because the design of SORT is essentially consisting of Kalman filter framework for predicting motion and Hungarian algorithm for data association while not dealing with the occlusions. Adding the object re-identification feature to deal with occlusions will weight significant complexities on the tracker. SORT is simple yet fast and maintains high accuracy with the deterministic output. Combined with fast YOLOv3 object detecor and SORT, it will be easy for us to run the experiment since both components are fast enough and produces the same result every time we run the same experiment.